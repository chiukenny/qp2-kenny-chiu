\documentclass[10pt]{article}
% header.tex
% this is where you load pacakges, specify custom formats, etc.

\usepackage[left=1in,right=1in,top=1in,footskip=25pt]{geometry} 
% \usepackage{changepage}
\usepackage{amsmath,amsthm,amssymb,amsfonts}
\usepackage{mathtools}
\usepackage{bm}
\usepackage{bbm}
\usepackage{mathrsfs}
\usepackage{accents}
\usepackage{xspace}
% enumitem for custom lists
\usepackage{enumitem}
% Load dsfont this to get proper indicator function (bold 1) with \mathds{1}:
\usepackage{dsfont}
\usepackage{centernot}

\usepackage[ruled,vlined,linesnumbered]{algorithm2e}
\usepackage{multirow}
\usepackage{booktabs}
\makeatletter
% Booktab style
\renewcommand*{\@algocf@pre@ruled}{\hrule height\heavyrulewidth depth0pt \kern\belowrulesep}
\renewcommand*{\algocf@caption@ruled}{\box\algocf@capbox\kern\aboverulesep\hrule height\lightrulewidth\kern\belowrulesep}
\renewcommand*{\@algocf@post@ruled}{\kern\aboverulesep\hrule height\heavyrulewidth\relax}
\makeatother

\usepackage[usenames,dvipsnames]{xcolor}

% set up commenting code (I will use this during marking)
\definecolor{CommentColor}{rgb}{0,.50,.50}
\newcounter{margincounter}
\newcommand{\displaycounter}{{\arabic{margincounter}}}
\newcommand{\incdisplaycounter}{{\stepcounter{margincounter}\arabic{margincounter}}}
\newcommand{\COMMENT}[1]{\textcolor{CommentColor}{$\,^{(\incdisplaycounter)}$}\marginpar{\scriptsize\textcolor{CommentColor}{ {\tiny $(\displaycounter)$} #1}}}

\usepackage{appendix}

% set up graphics
\usepackage{graphicx}
\DeclareGraphicsExtensions{.pdf,.png,.jpg}
\graphicspath{{fig/}}
\usepackage{float}

\usepackage[sorting=nyt,backend=biber,bibstyle=apa,citestyle=apa,giveninits=true]{biblatex}
\setlength\bibitemsep{1.5\itemsep}

\usepackage{fancyhdr}
\pagestyle{fancy}
\setlength{\headheight}{40pt}

%%%%%%%%%%%%%%%%%%%%%%%%%%%%%%%%%%%%%%%%%%%%%%%%%%%%%%%%%%%%%%%%%%%%%%%%%%%%%%%%%%%%
% most other packages you might use should be loaded before hyperref
%%%%%%%%%%%%%%%%%%%%%%%%%%%%%%%%%%%%%%%%%%%%%%%%%%%%%%%%%%%%%%%%%%%%%%%%%%%%%%%%%%%%

% Set up hyperlinks:
\definecolor{RefColor}{rgb}{0,0,.65}
\usepackage[colorlinks,linkcolor=RefColor,citecolor=RefColor,urlcolor=RefColor]{hyperref}

\usepackage[capitalize]{cleveref}
\crefname{appsec}{Appendix}{Appendices} % you can tell cleveref what to call things

\renewenvironment{abstract}
 {\par\noindent\textbf{\abstractname.}\ \ignorespaces}
 {\par\medskip}
% defs.tex
% this is where you define custom notation, commands, etc.


%%
% full alphabets of different styles
%%

% bf series
\def\bfA{\mathbf{A}}
\def\bfB{\mathbf{B}}
\def\bfC{\mathbf{C}}
\def\bfD{\mathbf{D}}
\def\bfE{\mathbf{E}}
\def\bfF{\mathbf{F}}
\def\bfG{\mathbf{G}}
\def\bfH{\mathbf{H}}
\def\bfI{\mathbf{I}}
\def\bfJ{\mathbf{J}}
\def\bfK{\mathbf{K}}
\def\bfL{\mathbf{L}}
\def\bfM{\mathbf{M}}
\def\bfN{\mathbf{N}}
\def\bfO{\mathbf{O}}
\def\bfP{\mathbf{P}}
\def\bfQ{\mathbf{Q}}
\def\bfR{\mathbf{R}}
\def\bfS{\mathbf{S}}
\def\bfT{\mathbf{T}}
\def\bfU{\mathbf{U}}
\def\bfV{\mathbf{V}}
\def\bfW{\mathbf{W}}
\def\bfX{\mathbf{X}}
\def\bfY{\mathbf{Y}}
\def\bfZ{\mathbf{Z}}

\def\bfb{\mathbf{b}}
\def\bfe{\mathbf{e}}
\def\bft{\mathbf{t}}
\def\bfy{\mathbf{y}}

\def\bfom{\bm{\omega}}
\def\bfOm{\bm{\Omega}}

% bb series
\def\bbA{\mathbb{A}}
\def\bbB{\mathbb{B}}
\def\bbC{\mathbb{C}}
\def\bbD{\mathbb{D}}
\def\bbE{\mathbb{E}}
\def\bbF{\mathbb{F}}
\def\bbG{\mathbb{G}}
\def\bbH{\mathbb{H}}
\def\bbI{\mathbb{I}}
\def\bbJ{\mathbb{J}}
\def\bbK{\mathbb{K}}
\def\bbL{\mathbb{L}}
\def\bbM{\mathbb{M}}
\def\bbN{\mathbb{N}}
\def\bbO{\mathbb{O}}
\def\bbP{\mathbb{P}}
\def\bbQ{\mathbb{Q}}
\def\bbR{\mathbb{R}}
\def\bbS{\mathbb{S}}
\def\bbT{\mathbb{T}}
\def\bbU{\mathbb{U}}
\def\bbV{\mathbb{V}}
\def\bbW{\mathbb{W}}
\def\bbX{\mathbb{X}}
\def\bbY{\mathbb{Y}}
\def\bbZ{\mathbb{Z}}

% cal series
\def\calA{\mathcal{A}}
\def\calB{\mathcal{B}}
\def\calC{\mathcal{C}}
\def\calD{\mathcal{D}}
\def\calE{\mathcal{E}}
\def\calF{\mathcal{F}}
\def\calG{\mathcal{G}}
\def\calH{\mathcal{H}}
\def\calI{\mathcal{I}}
\def\calJ{\mathcal{J}}
\def\calK{\mathcal{K}}
\def\calL{\mathcal{L}}
\def\calM{\mathcal{M}}
\def\calN{\mathcal{N}}
\def\calO{\mathcal{O}}
\def\calP{\mathcal{P}}
\def\calQ{\mathcal{Q}}
\def\calR{\mathcal{R}}
\def\calS{\mathcal{S}}
\def\calT{\mathcal{T}}
\def\calU{\mathcal{U}}
\def\calV{\mathcal{V}}
\def\calW{\mathcal{W}}
\def\calX{\mathcal{X}}
\def\calY{\mathcal{Y}}
\def\calZ{\mathcal{Z}}

\def\tildeU{\widetilde{U}}
\def\hatb{\widehat{\bfb}}
\def\hatOm{\widehat{\bfOm}}
\def\hatW{\widehat{\bfW}}
\def\hatmu{\widehat{\mu}}


%%%%%%%%%%%%%%%%%%%%%%%%%%%%%%%%%%%%%%%%%%%%%%%%%%%%%%%%%%
% text short-cuts
\def\iid{i.i.d.\ } %i.i.d.
\def\ie{i.e.\ }
\def\eg{e.g.\ }
\def\Polya{P\'{o}lya\ }
%%%%%%%%%%%%%%%%%%%%%%%%%%%%%%%%%%%%%%%%%%%%%%%%%%%%%%%%%%

%%%%%%%%%%%%%%%%%%%%%%%%%%%%%%%%%%%%%%%%%%%%%%%%%%%%%%%%%%
% quasi-universal probabilistic and mathematical notation
% my preferences (modulo publication conventions, and clashes like random vectors):
%   vectors: bold, lowercase
%   matrices: bold, uppercase
%   operators: blackboard (e.g., \mathbb{E}), uppercase
%   sets, spaces: calligraphic, uppercase
%   random variables: normal font, uppercase
%   deterministic quantities: normal font, lowercase
%%%%%%%%%%%%%%%%%%%%%%%%%%%%%%%%%%%%%%%%%%%%%%%%%%%%%%%%%%

% operators
\def\P{\bbP} %fundamental probability
\def\E{\bbE} %expectation
% conditional expectation
\DeclarePairedDelimiterX\bigCond[2]{[}{]}{#1 \;\delimsize\vert\; #2}
\newcommand{\conditional}[3][]{\bbE_{#1}\bigCond*{#2}{#3}}
\def\Law{\mathcal{L}} %law; this is by convention in the literature
\def\indicator{\mathds{1}} % indicator function

% sets and groups
\def\borel{\calB} %Borel sets
\def\sigAlg{\calA} %sigma-algebra
\def\filtration{\calF} %filtration
\def\grp{\calG} %group

% binary relations
\def\condind{{\perp\!\!\!\perp}} %independence/conditional independence
\def\equdist{\stackrel{\text{\rm\tiny d}}{=}} %equal in distribution
\def\equas{\stackrel{\text{\rm\tiny a.s.}}{=}} %euqal amost surely
\def\simiid{\sim_{\mbox{\tiny iid}}} %sampled i.i.d

% common vectors and matrices
\def\onevec{\mathbf{1}}
\def\iden{\mathbf{I}} % identity matrix
\def\supp{\text{\rm supp}}

% misc
% floor and ceiling
\DeclarePairedDelimiter{\ceilpair}{\lceil}{\rceil}
\DeclarePairedDelimiter{\floor}{\lfloor}{\rfloor}
\newcommand{\argdot}{{\,\vcenter{\hbox{\tiny$\bullet$}}\,}} %generic argument dot

\DeclareMathOperator*{\argmax}{arg\,max}
\DeclareMathOperator*{\argmin}{arg\,min}
%%%%%%%%%%%%%%%%%%%%%%%%%%%%%%%%%%%%%%%%%%%%%%%%%%%%%%%%%%

%%%%%%%%%%%%%%%%%%%%%%%%%%%%%%%%%%%%%%%%%%%%%%%%%%%%%%%%%%
%% some distributions
% continuous
\def\UnifDist{\text{\rm Unif}}
\def\BetaDist{\text{\rm Beta}}
\def\ExpDist{\text{\rm Exp}}
\def\GammaDist{\text{\rm Gamma}}
% \def\GenGammaDist{\text{\rm GGa}} %Generalized Gamma

% discrete
\def\BernDist{\text{\rm Bernoulli}}
\def\BinomDist{\text{\rm Binomial}}
\def\PoissonPlus{\text{\rm Poisson}_{+}}
\def\PoissonDist{\text{\rm Poisson}}
\def\NBPlus{\text{\rm NB}_{+}}
\def\NBDist{\text{\rm NB}}
\def\GeomDist{\text{\rm Geom}}
% \def\CRP{\text{\rm CRP}}
% \def\EGP{\text{\rm EGP}}
% \def\MittagLeffler{\text{\rm ML}}
%%%%%%%%%%%%%%%%%%%%%%%%%%%%%%%%%%%%%%%%%%%%%%%%%%%%%%%%%%

%%%%%%%%%%%%%%%%%%%%%%%%%%%%%%%%%%%%%%%%%%%%%%%%%%%%%%%%%%
% Project-specific notation should go here
% (Because it's at the end of the file, it can overwrite anything that came before.)

%e.g.,
\def\Laplacian{\calL}
%\def\P{\calP}

% combinatorial objects
\def\perm{\sigma} %fixed permutation
\def\Perm{\Sigma} %random permutation
\def\part{\pi} %fixed partition
\def\Part{\Pi} %random partition

% Kernels
\def\MMD{\mathrm{MMD}}
\def\hatMMD{\widehat{\MMD}}
\def\dhatMMD{\widehat{\vphantom{\rule{1.5pt}{5.5pt}}\smash{\hatMMD}}}
\def\d{\mathrm{d}}

\def\xo{x^{(1)}}
\def\xt{x^{(2)}}
\def\yo{y^{(1)}}
\def\yt{y^{(2)}}
\def\go{g^{(1)}}
\def\gtw{g^{(2)}}

% Theorems
\newtheorem{theorem}{Theorem}[section]
\newtheorem{proposition}[theorem]{Proposition}
\newtheorem{corollary}[theorem]{Corollary}
\renewcommand{\thetheorem}{\arabic{theorem}}


\newcommand{\mre}{\textsc{MRE}\xspace}
\newcommand{\Var}{\mathrm{Var}}
\newcommand{\bias}{\mathrm{bias}}
\newcommand{\barY}{\bar{Y}}

\newcommand{\T}{\top}

\newcommand\sbullet[1][.5]{\mathbin{\vcenter{\hbox{\scalebox{.75}{$\bullet$}}}}}

%%%%%%%%%%%%%%%%%%%%%%%%%%%%%%%%%%%%%%%%%%%%%%%%%%%%%%%%%%

\newcommand{\todo}{{\color{red}{TODO}}}

\title{Stepped Wedge Cluster Randomized Trials\\\vspace{0.5em}\large{STAT 548 Qualifying Paper}}
\author{Kenny Chiu}
\date{\today}

\bibliography{refs.bib}

\begin{document}

\maketitle

\vspace{2em}

\begin{abstract}
\todo
\end{abstract}

\vspace{2em}


\section{Introduction}

The work of \textcite{Hussey:2007} can be viewed as an entry-point to the study of stepped wedge cluster randomized trials (SW-CRT), which is a type of experimental design that is particularly pragmatic compared to alternative designs that may also be used in similar contexts. In this report, we review the paper by \citeauthor{Hussey:2007}. We summarize the main ideas while filling in missing details, replicate one of the empirical studies with a slight modification, and provide a critique of the paper. We also discuss how the literature on SW-CRTs has developed since the paper was published, and highlight some of the common extensions to the standard model presented in the paper.
\\

This report is organized as follows: Section~\ref{sec:summary} summarizes the paper and provides additional details where we feel was missed in the original work; Section~\ref{sec:critique} discusses our perspective and critique of the paper; Section~\ref{sec:simulation} describes our attempt to replicate the paper's simulation study and our findings; Section~\ref{sec:extension} presents modern common extensions to the standard SW-CRT model; and Section~\ref{sec:discussion} concludes this report by summarizing our discussion. Appendix~\ref{apx} includes our longer explanations and derivations from Section~\ref{sec:summary} to avoid disrupting the flow of the summary.


\section{Summary and additional details} \label{sec:summary}

In this section, we summarize the main ideas of the paper by \textcite{Hussey:2007} and provide additional details that we feel are missing from the paper. Our longer explanations and derivations are included in Appendix~\ref{apx} to avoid disrupting the flow of the summary.

\subsection{Context and motivation}

Cluster randomized trials (CRT) are characterized by the randomization to interventions being done at the group or cluster-level rather than at the individual-level, and it is typically assumed that the individuals within a cluster are correlated. CRTs are considered when it is not convenient or not appropriate to administer an intervention to single individuals. \textcite{Hussey:2007} comment that the majority of CRT designs studied and employed (at the time of the paper) featured parallel designs where approximately half of the clusters are simultaneously given one intervention and the other half are simultaneously given another. While these parallel CRTs are convenient analytically, they may present problems in practice if, for example, there are logistical constraints that make delivering the intervention simultaneously across multiple clusters difficult. Other potential issues of parallel designs include ethical concerns where if there is an expectation that a new intervention improves on an existing one, then withholding the new intervention from certain clusters would be problematic. Therefore, the main objective of \citeauthor{Hussey:2007}'s work is to promote the stepped wedge CRT design as an alternative that addresses the potential issues of the parallel design, and to provide an overview of how the data collected from such a design are analyzed. In addition, \citeauthor{Hussey:2007} also discuss certain statistical considerations of SW-CRT designs, such as power and efficiency of estimators, and how these properties are affected by model assumptions and design parameters.

\subsection{SW-CRT design}

The SW-CRT design is a type of crossover design. However, unlike in standard crossover CRTs where clusters start with possibly different treatments and switches treatments at a determined time point, SW-CRT are characterized by
\begin{enumerate}

\item
the crossover being unidirectional where all clusters start with the same treatment (the control or an existing treatment) and end with the same treatment (the intervention), and

\item
the staggered times at which each cluster switches to the intervention (with the times being randomized across clusters).

\end{enumerate}
Figure~\ref{fig:designs} from the paper clearly illustrates the differences between the discussed CRT designs.

\begin{figure}[H]
\centering
\begin{tabular}{@{}rr|c@{}}
\multicolumn{2}{l|}{\multirow{2}{*}{\textbf{Parallel}}} & \multicolumn{1}{r}{Time} \\
& & 1 \\
\hline
\multirow{4}{*}{Cluster} & 1 & 1 \\
& 2 & 1 \\
& 3 & 0 \\
& 4 & 0
\end{tabular}
\qquad
\begin{tabular}{@{}rr|cc@{}}
\multicolumn{2}{l|}{\multirow{2}{*}{\textbf{Crossover}}} & \multicolumn{2}{r}{Time} \\
& & 1 & 2 \\
\hline
\multirow{4}{*}{Cluster} & 1 & 1 & 0 \\
& 2 & 1 & 0 \\
& 3 & 0 & 1\\
& 4 & 0 & 1
\end{tabular}
\qquad
\begin{tabular}{@{}rr|ccccc@{}}
\multicolumn{2}{l|}{\textbf{Stepped}} & \multicolumn{5}{r}{Time} \\
 \multicolumn{2}{l|}{\textbf{wedge}} & 1 & 2 & 3 & 4 & 5 \\
\hline
\multirow{4}{*}{Cluster} & 1 & 0 & 1 & 1 & 1 & 1 \\
& 2 & 0 & 0 & 1 & 1 & 1 \\
& 3 & 0 & 0 & 0 & 1 & 1 \\
& 4 & 0 & 0 & 0 & 0 & 1
\end{tabular}
\caption{Example treatment schedules for parallel, crossover, and stepped wedge CRT designs. The control/existing treatment and the intervention are denoted 0 and 1, respectively. Figure slightly modified from \parencite{Hussey:2007}.}
\label{fig:designs}
\end{figure}

From Figure~\ref{fig:designs}, it can be seen how the SW-CRT design addresses the practical issues of the parallel design. Rather than simultaneously delivering the intervention to multiple groups, SW-CRTs stagger the delivery to clusters across different times, potentially alleviating logistical concerns. Furthermore, all clusters eventually obtain the intervention, which avoids the problem of withholding the intervention from certain clusters. The SW-CRT design is not without its own complications, however. Staggering the times means that the duration of the study is elongated relative to the parallel and crossover designs. The unidirectional crossover also implies that time may be correlated with the effect of the intervention, which may lead to issues in estimation of the intervention effect when analyzing the data.
\\

Beyond the general characteristics of SW-CRTs described above, other considerations and design parameters (e.g., cluster sizes, number of clusters crossing over at each time point, new individuals in a cluster across time, etc.) will depend on the context of the specific study. \citeauthor{Hussey:2007} examine the model for a SW-CRT in detail under a specific setting. How the model changes to varying study contexts are only briefly mentioned or, in the case of some variations, not discussed at all. We return to this point in our critique of the paper in Section~\ref{sec:critique} and again when we discuss model extensions in Section~\ref{sec:extension}.

\subsection{Assumed setting and SW-CRT model}

The SW-CRT model that \textcite{Hussey:2007} examine in their paper can be considered the ``basic'' or ``standard'' model and is based on a particular example trial.

\subsubsection{Expedited Partner Treatment trial}

The primary SW-CRT setting that \citeauthor{Hussey:2007} work under is based on the context of the 2012 \textit{Washington State Community Expedited Partner Treatment (EPT) Trial}. The hypothesis of interest in this study was whether a public health program that increases the use of EPT decreases the prevalence of chlamydia in young women and the incidence of gonorrhea in WA state. The intervention---promotion of EPT and targeted provision of partner services---was instituted in 23 WA state local health jurisdictions (LHJ) across four waves separated by 6--9 months. Each wave included approximately six LHJs, and the order in which LHJs initiated the intervention were randomly assigned. The measured primary outcomes in the study included the prevalence of chlamydia in women aged 15--25 who tested positive in participating clinics and the incidence of gonorrhea in women as ascertained through public health reporting.
\\

One of the key design aspects to consider for modeling is how the clusters and the individuals are defined. For example, in the chlamydia study of the EPT trial, the individual LHJs are the clusters, and the individuals in a cluster are the women who were tested in a participating clinic during a particular timeframe. It is important to note that the women within a single cluster differ at different time points over the trial (i.e., the study uses a cross-sectional CRT). While the number of women who were tested varied between LHJs and across time within a single LHJ, there were likely more than enough women tested that only a sample of women for each time point and for each LHJ were sufficient for fitting a reasonable model. In such a scenario, it is analytically more convenient to take samples of equal sizes across LHJs and across time.
\\

Another aspect to consider is the form of the outcome. In the EPT trial, the measured outcomes are prevelance of chlamydia and incidence of gonorrhea, both of which are continuous measures. Thus, based on the EPT trial, \citeauthor{Hussey:2007} focus on the properties of a continuous outcome SW-CRT model that assumes a cross-sectional design and fixed cluster sizes.

\subsubsection{SW-CRT model} \label{sec:model}

A linear mixed effects model (LMM) can be used to model the SW-CRT design described under the setting in the previous section. Assuming that there are $I$ clusters, $T$ time points, and $N$ individuals in each cluster at each time point, we can define the mean for cluster $i$ at time $j$ as
\[
\mu_{ij} = \mu + \alpha_i + \beta_j + X_{ij}\theta
\]
where
\begin{itemize}

\item
$\mu$ is the overall mean across clusters and time,

\item
$\alpha_i\sim N(0,\tau^2)$ is a random effect for cluster $i\in\{1,\ldots,I\}$ that captures the correlation between individuals,

\item
$\beta_j$ is a fixed effect for time point $j\in\{1,\ldots,T-1\}$ (assuming $\beta_T=0$ for identifiability),

\item
$X_{ij}$ is a treatment indicator for cluster $i$ at time $j$ with 1 denoting the intervention, and

\item
$\theta$ is the treatment effect of interest.

\end{itemize}
Using a slightly different notation from \citeauthor{Hussey:2007}, a model at the individual-level is then given by
\[
Y_{ijk} = \mu_{ij} + e_{ijk}
\]
where $e_{ijk}\sim N(0,\sigma^2)$ are i.i.d. noise, and this leads to a model at the cluster-level given by
\[
\barY_{ij} = \mu_{ij} + \frac{1}{N}\sum_{k=1}^Ne_{ijk} \;.
\]
As mentioned, this model can be considered the standard model for a SW-CRT design as the assumptions it makes are fairly basic. The model can be extended many ways depending on the setting of the specific study. We revisit this point in Section~\ref{sec:extension}.
\\

Aside from the model itself, two other quantities obtained from the model are also commonly referred to in the CRT literature. The intraclass correlation (ICC) $\rho=\frac{\tau^2}{\tau^2+\sigma^2}$ (and its induced variation inflation factor $1+(N-1)\rho$) and the coefficient of variation (CV) $\frac{\tau}{\mu}$ characterize the effect of the within-cluster correlation on the cluster mean variance and are often the parameters being adjusted in CRT simulation studies. We provide some intuition for these quantities in Appendix~\ref{apx:correlation}.


\subsection{Methods and analysis}

\textcite{Hussey:2007} discuss several ideas related to estimation and analysis of the simple model for a SW-CRT. We highlight the key points in this section. We also note that \citeauthor{Hussey:2007} generally skip over explanations and derivations when discussing an estimator or one its statistical properties. We try to address these gaps in this section and in Appendix~\ref{apx}.

\subsubsection{Estimation of the treatment effect}

The general objective when analyzing data from a SW-CRT is to estimate and test the treatment effect $\theta$. When the variance components $\tau^2$ and $\sigma^2$ are known, a cluster-level estimation of $\theta$ is possible using weighted least squares (WLS). While this approach is useful for conducting a pre-trial power analysis, it is generally the case in practice that $\tau^2$ and $\sigma^2$ are unknown. When the variance components are unknown, an individual-level analysis using generalized linear mixed effects models (GLMM) or generalized estimating equations (GEE) will likely be the preferred approach. \citeauthor{Hussey:2007} caution that the LMM, GLMM, and GEE approaches all rely on asymptotic results, and so an analysis of a SW-CRT involving few clusters or time points may produce misleading findings.
\\

\citeauthor{Hussey:2007} note that when there are no time effects on the outcome (i.e., when $\beta_j=0$ for all~$j$), estimation of the treatment effect $\theta$ can be done using a within-cluster analysis (an analysis based on comparing the control and intervention time periods for each cluster). This case also allows testing of the treatment effect using a paired t-test where the two groups correspond to the control and the intervention time periods. However, if it is incorrectly assumed that the time effects are trivial, the estimator for $\theta$ will be biased. We provide additional details about the estimator and the bias in Appendix~\ref{apx:estimator}.
\\

\citeauthor{Hussey:2007} also discuss the relative efficiency of the WLS estimator of the treatment effect compared to the within-cluster estimator. When there are no time effects, the WLS estimator is always more efficient than the within-cluster estimator. When there are time effects, the within-cluster estimator is more efficient (but likely biased). Note that there is an error in the efficiency given by \textcite{Hussey:2007} as pointed out by \textcite{Liao:2015}, but the above statements still hold. \textcite{Hussey:2007} also imply that an exception to the WLS estimator being more efficient in the no time effect-case is when $\tau^2=0$, but the efficiency proof by \textcite{Liao:2015} appears to hold regardless. We provide further details for the efficiency and the proof for the no time effect-case in Appendix~\ref{apx:efficiency}.

\subsubsection{Power and relevant factors}

To obtain an approximate power for the study, \textcite{Hussey:2007} prescribe using a Wald test to test the hypothesis $H_0:\theta=0$ against a simple alternative $H_a: \theta=\theta_a$. The power for a two-tailed test of size $\alpha$ given by \textcite{Hussey:2007} is
\[
\text{power} \approx \Phi\left(\frac{\theta_a}{\sqrt{\Var(\hat{\theta})}} - Z_{1-\frac{\alpha}{2}}\right)
\]
where $\Phi$ is the cumulative distribution function of the standard normal and $Z_{1-\frac{\alpha}{2}}$ is the $\left(1-\frac{\alpha}{2}\right)$-th quantile of the standard normal. We note that this approximation implicitly makes the assumption that $\theta_a>0$ and is not too small. This approach of computing power is applicable to any estimator that is normally distributed or based on large-sample statistics. We provide additional details about the Wald test and the power calculation in Appendix~\ref{apx:wald}.
\\

\citeauthor{Hussey:2007} also discuss how the power decreases with fewer measured time points and with delays in the treatment effect. Both of these factors are more of a concern when there are constraints on the duration of the trial. The decrease in power can be mitigated by allowing for more time points (i.e., fewer clusters randomized to each time point) and by increasing the length of each time interval to allow for the treatment effect to realize over fewer time points. We provide an explanation for these statements in Appendix~\ref{apx:power}.

\subsection{Simulation study of analysis methods}

\textcite{Hussey:2007} performed a simulation study to compare the power of the test $H_0:\theta=0$ versus $H_a:\theta\neq0$ in a SW-CRT analysis using LMM, GEE, and GLMM. The case of equal cluster sizes and the case of unequal cluster sizes were both investigated. Their results suggest that LMM has greater power than the other two methods when the cluster sizes are equal, and otherwise GEE and GLMM have greater power. They explain that when the cluster sizes are unequal, the clusters need to be weighted in LMM, but the weights will depend on the true (unknown) variance components.
\\

We comment that certain details of the simulation study are ambiguous based on \citeauthor{Hussey:2007}' description. We try to clarify these details through our replication of their study in Section~\ref{sec:simulation}.


\section{Critical appraisal} \label{sec:critique}

We comment on our perspective of the paper by \textcite{Hussey:2007} and critique the paper's strengths, limitations and weaknesses.
\\

From our understanding of the context of the paper, the paper is clearly aimed at addressing a common problem (the use of parallel designs in CRTs even in the presence of logistical or other concerns) and making a case for the SW-CRT design as a solution to the problem. While the design itself was not new, it is said that the design was infrequently employed in CRTs at the time of the paper and that analyses of such designs generally varied from trial to trial \parencite{Brown:2006}. From this perspective, the main contribution of \citeauthor{Hussey:2007} is the overview of SW-CRTs detailing the motivation, design, analysis, statistical properties, and possible extensions/issues of consideration that is presented in a relatively succinct and accessible format. Looking at the number of citations that the paper has (approximately 970 at the time of this report) and how the literature has developed since then, the paper successfully achieved its purpose of promoting SW-CRT designs and being an entry-point for those unfamiliar with the design.
\\

The main limitation of the paper is the limited breadth and minimal depth it provides on the technical details of the SW-CRT. This limitation was likely the tradeoff on keeping the paper accessible, which would be an important consideration of the authors as those who actually implement and would be interested in such trials are mostly limited to policy makers and researchers who may have limited statistical background. For this reason, the paper focuses mainly on the technical aspects that are of practical relevance such as analysis and power. Even then, the technical discussion is restricted to primarily the basic model of focus. Derivations are skipped over entirely, and the discussion of extended models are generally left to other references.
\\

From our own attempt at replicating their simulation study, we found that certain details of the simulation are unclear from \citeauthor{Hussey:2007}' description. For example, it is mentioned that the powers in the results are calculated using the standard variance. It may be reasonable to assume that the standard variance refers to the standard error returned in the output of the used function. However, this estimate would be only approximate in the case of LMM where binary data are aggregated for cluster-level analysis, and whether this is the case in the simulation is left unaddressed. Other details of the simulation similarly are missing and must be inferred from the context.
\\

In our opinion, the weaknesses of the paper mainly relate to its presentation and its organization. From the writing itself, it is not entirely obvious which aspects of the paper are novel contributions. For example, the LMM is presented in the paper as a common model in CRTs, but the paper is often cited in the literature when the LMM is referenced \parencite[e.g.,][]{Harrison:2020,Bowden:2021,DavisPlourde:2021}. It can be unclear from the presentation of the paper whether the LMM section is introducing background or a proposed model. Our understanding is that most of the content are synthesized overviews of existing ideas with the exception of the power calculation procedure, the factors affecting power, and the simulation study. Related to the presentation, the organization of the paper can also be a source of confusion as the statistical issues subsections seemingly jump from one idea to another (sometimes under disconnecting assumptions, e.g., when no time effects assumed in one section but not the following). Finally, there are also a number of typos throughout the paper. While most are minor where the intended idea can be inferred from a second glance, one major typo (or error) is the missing factor in the denominator of the relative efficiency of the estimators as discovered by \textcite{Liao:2015}. As the derivation is not included with the paper, this error would be difficult to notice otherwise.
\\

As part of the objectives for this report, we address the main limitation of the paper by developing the discussion on breadth and depth of SW-CRT designs. Our summary of the paper in Section~\ref{sec:summary} included notes of where we determined technical details could be further developed with our added details given in Appendix~\ref{apx}. The following section describes our attempts to replicate the simulation study and what we found as a result. Section~\ref{sec:extension} addresses the breadth of extensions to the standard SW-CRT model that are now commonly seen in modern SW-CRT literature.


\section{Simulation investigation} \label{sec:simulation}

In this section, we describe our attempt to replicate the simulation by \textcite{Hussey:2007} and to clarify the missing or ambiguous details. The motivation of their simulation study was to compare the power for testing the hypothesis $H_0:\theta=0$ between analyses using LMM, GEE and GLMM. The data used in the study was simulated based on the EPT trial under two conditions (equal and unequal cluster sizes). Their results are reproduced in Table~\ref{tab:original} for convenience of comparison.

\begin{table}[H]
\centering
\begin{tabular}{@{}lclllclll@{}}
\toprule
\multirow{2}{*}{\parbox{0.75cm}{Risk ratio}} && \multicolumn{3}{l}{Same cluster sizes} && \multicolumn{3}{l}{Different cluster sizes} \\
\cmidrule{3-5}\cmidrule{7-9}
 & & LMM & GEE & GLMM && LMM & GEE & GLMM \\
\midrule
1.0 &&  0.056 (0.057) & 0.084 (0.052) & 0.076 (0.053) && 0.048 (0.038) & 0.095 (0.053) & 0.069 (0.049) \\
0.7 && 0.697 (0.658) & 0.719 (0.644) & 0.716 (0.580) && 0.307 (0.307) & 0.703 (0.577) & 0.697 (0.559) \\
0.6 && 0.907 (0.884) & 0.907 (0.866) & 0.917 (0.820) && 0.487 (0.503) & 0.879 (0.807) & 0.906 (0.805) \\
0.5 && 0.988 (0.984) & 0.990 (0.981) & 0.992 (0.948) && 0.625 (0.653) & 0.982 (0.946) & 0.986 (0.942) \\
\bottomrule
\end{tabular}
\caption{The original results of the simulation study by \textcite{Hussey:2007}. The power is computed using both the ``standard variance'' and a jackknife estimate of the variance (in parentheses).}
\label{tab:original}
\end{table}

\subsection{Simulating the data}

We explain how we simulated the data in our attempt to replicate the study. In each of the 1000 simulations, the data is simulated based on the EPT trial in which
\begin{itemize}

\item
the number of clusters is $I=24$,

\item
the number of measured time points is $T=5$ (four randomization times with six clusters crossing over at each),

\item
the baseline prevalence of disease is $\mu=0.05$, and

\item
the between-cluster variance is assumed to be $\tau^2=0.000225$.

\end{itemize}
In the equal size case, the number of individuals in each cluster at each time point is $N=100$. In the unequal size case, for each simulation, the parameter of a multinomial distribution is sampled from a Dirichlet distribution with parameters $(1,\ldots,1)$. A multinomial distribution with this parameter is then used to randomly assign $(N-2)I=2352$ individuals across $I$ clusters. Two additional individuals are added to each cluster afterwards (to avoid empty clusters, and two because we found clusters of size one sometimes resulted in singular convergence during the fitting of the model), making the total number of individuals across clusters equal to $NI=2400$. Each cluster keeps the same number of individuals across time points.
\\

Once the cluster sizes are determined in a simulation, individual-level data is simulated for each cluster at each time point. Each individual is simulated from a Bernoulli distribution with probability $p$, where $p$ is given by
\[
p = \max(0,\mu + \alpha_i + X_{ij}\theta) \;.
\]
The cluster random effects $\alpha_i$, $i\in\{1,\ldots,I\}$, are sampled from a Normal$(0,\tau^2)$ distribution and are resampled in every simulation. The treatment indicator $X_{ij}$ for each time point $j\in\{1,\ldots,T\}$ is determined after each cluster is randomly assigned a crossover time, which are also reshuffled in every simulation. The treatment effect $\theta$ is determined by the risk ratio (RR) chosen for the study where $\theta=\mu(\text{RR}-1)$. The values of RR examined by \citeauthor{Hussey:2007} include $\{0.5,0.6,0.7,1\}$, with $\theta=0$ for $\text{RR}=1$ and $\theta=-0.025$ for $\text{RR}=0.5$. The max is taken for $p$ in case $p < 0$ due to the sampled cluster effect. It is assumed that there are no time effects.
\\

The simulated dataset in each simulation has $12,\!000$ rows. The same 1000 datasets are used to obtain the results for all approaches.

\subsection{Fitting the model and calculating the power}

To fit the models in the LMM, GEE and GLMM approaches, we use the same \verb|R| functions specified by \citeauthor{Hussey:2007}. For LMM, the individual-level data is first aggregated into a dataset with $120$ rows where each row corresponds to a particular cluster and time point. The \verb|lme()| function from the \verb|R| package \verb|nlme| is then used to fit the LMM on the cluster-time dataset. For GEE and GLMM, the \verb|gee()| function from the \verb|gee| package and the \verb|glmmPQL()| function from the \verb|MASS| package are used, respectively, with the individual-level dataset. For GEE, the correlation structure is specified to be exchangeable.
\\

In all approaches, the fitted model for the mean response includes a fixed intercept $\hat{\mu}$, a fixed treatment effect $\hat{\theta}$, and random effects $\hat{\alpha}_i$. We also include fixed time effects $\hat{\beta}_j$ in the model, and we will show how the results differ when time is excluded as a variable from the model.
\\

To approximate the power, we compute in each simulation the Wald test statistic given by
\[
W=\frac{\hat{\theta}}{\sqrt{\Var(\hat{\theta})}}
\]
where $\sqrt{\Var(\hat{\theta})}$ is either taken as the standard error of $\hat{\theta}$ given in the outputs of the functions (which we assume is what \citeauthor{Hussey:2007} refer to as the ``standard variance'') or by a jackknife estimate. \todo: robust SE. The jackknife estimate is computed by \todo. We reject the null hypothesis if $|W|>z(0.975)$ where $z(0.975)$ is the 97.5\% quantile of the standard normal distribution. The power is then estimated by the proportion of rejects out of the 1000 simulations. \todo t test?

\subsection{Results}

\subsection{Discussion}


\section{Extensions of the standard model} \label{sec:extension}

The model examined by \textcite{Hussey:2007} can be considered the basic model as it makes relatively simple assumptions about the design and the effects. Modern literature has extended the basic model in various ways that better align with the conditions that may be seen in practice. We present the common extensions to the model and discuss what they aim to address in this section.

\subsection{Unequal cluster sizes}

The case of unequal cluster sizes is one of the few variations of SW-CRTs that \citeauthor{Hussey:2007} discuss in their paper. The basic model assumes that all of the clusters have the same size, and this may have been a reasonable assumption for the EPT trial as there were likely more than enough participants in each LHJ during each time period. However, other trials may have a small population of study and may have difficulty achieving even the desired cluster size. In this case, sizes may vary between not only clusters but potentially also between time periods within a single cluster. A review of SW-CRTs published up to March 2016 found that almost half of the considered trials involved unequal cluster sizes \parencite{Kristunas:2017}.
\\

Unequal cluster sizes does not change the basic model but leads to potentially differing variances across cluster means (given by $\Var(\bar{y}_{ij.})=\tau^2+\frac{\sigma^2}{N_{ij}}$). This affects cluster-level estimation of the model parameters as the correct weights in WLS depend on the true, unknown variance components. This consequently affects the power for testing the treatment effect, and \textcite{Hussey:2007} found that unequal cluster sizes resulted in the LMM method having weaker power compared to the GLMM and GEE methods in their simulation study.
\\

Examples of recent literature that studied the impact of unequal cluster sizes on SW-CRT analysis include the works by \textcite{Martin:2019}, \textcite{Ouyang:2020}, \textcite{Kasza:2021}, and \textcite{Tian:2021}. While these studies all examine differing trial contexts, the general findings echo that of \citeauthor{Hussey:2007}' in that unequal cluster sizes affect the power (and efficiency of estimators), and that using methods that account for the unequal cluster sizes may partially recover some of the lost power (efficiency).

\subsection{Delayed treatment effect}

The case where there are delays in the treatment effect is also one of the variations that \citeauthor{Hussey:2007} discuss. The basic model assumes that once the intervention is applied to a cluster, its full effect on the measured response is immediate and visible. In reality, depending on what the intervention is, its effect on the response may only be partial in the short term and may take a longer time period in order for its full effect to manifest. Examples of trials where a delayed effect may be expected include the implementation of infant Hepatitis B vaccination programs for reducing incidences in the population \parencite{Hughes:2015}, and the use of repellents for reducing malaria in villages \parencite{Agius:2020}.
\\

It is notable that one can modeling delayed treatment effects by allowing the time intervals in the SW-CRT to be long enough for the full treatment effect to be applied. However, if logistical constraints make extending the duration of the trial infeasible, then delayed effects can be modeled by modifying the parameters of the basic model. The most general delayed effects model takes the treatment indicator $X_{ij}$ to be in $[0,1]$ rather than as binary, in which case the treatment effect $\theta$ is then interpreted as the full effect. Further complications can be considered in the model, such as the length of the delay and whether the delay is known or needs to be estimated. These considerations would generally affect the model parameterization.
\\

\textcite{Hussey:2007} empirically show that a delayed effect reduces the power (and we show this theoretically in Appendix~\ref{apx:delay} using the within-cluster estimator as an example). While delayed treatment effects are a common point of consideration in the literature (in both the analysis of real trials and the development of SW-CRT theory), there does not appear to be much recent work where the primary objective is dedicated to studying its effect.

\subsection{Non-normal response}

The basic model assumes that the individual-level responses (and therefore the cluster-level responses) are normally distributed, but it may be the case that the individual-level responses are non-normal. \citeauthor{Hussey:2007} examine the EPT trial where the individual-level responses are binary (though their model assumes normal responses), and the responses may be generalized to distributions in the exponential family by using a GLMM fitted with a particular link function. An example of a SW-CRT trial that involved non-normal outcomes is the DECIDE-LVAD trial that measured the effect of educational materials on patient decision making using a 10-point Likert scale \parencite{Allen:2018}.
\\

As mentioned, the general approach for extending the basic model to non-normal responses is to use a GLMM with a link function $g$, i.e., to use the model
\[
g(\bbE[Y_{ijk}]) = \mu_{ij}
\]
where the link function $g$ is chosen appropriately for the desired distribution of the response. For simple outcomes (e.g., the binary response in the EPT trial), cluster-level analyses may still be possible and meaningful. However, more complex outcomes will likely require individual-level analyses. Therefore, SW-CRTs featuring a non-normal response will generally use GLMM or GEE for fitting the model.
\\

A recent review paper commented that there is still much room for development of SW-CRTs with non-normal responses in the literature \parencite{Li:2021}. Recent work have proposed formulas for power and sample size calculations in SW-CRT designs with binary outcomes \parencite{Wang:2021} and discrete outcomes \parencite{Xia:2021}. From our perspective, it appears that there are still gaps with this topic in the literature. In particular, non-normal and non-discrete responses (e.g., survival times) in SW-CRT designs seem to be one direction that has not been explored much in the literature.

\subsection{Cohort designs}

The EPT trial is an example of a cross-sectional SW-CRT where the participants in a cluster at a particular time point are different from the participants in the same cluster at another time point. A cohort SW-CRT design differs from the cross-sectional design in that a cluster tracks the same participants over multiple time points. An open cohort design (e.g., ongoing recruitment of participants) tracks the same participants over a partial duration of the trial, while a closed cohort design tracks the same participants over the entire duration of the trial (except in the case of dropouts). An example of a cohort SW-CRT is the INSTTEPP trial that investigated if a boot camp translation process for self-management support (1) improved outcomes in chronic patients and (2) improved clinician and staff attitudes toward self-management \parencite{Nease:2018}. The trial employed an open cohort design for the patients (where new patients may join the study part-way through) and a closed cohort design for the clinicians and staff (with the same participants being evaluated throughout the study).
\\

As cohort designs imply potential repeated measurements for each participant, the general approach for extending the basic model to account for this is to add individual-level random effects, i.e.,
\[
Y_{ijk} = \mu + \alpha_i + \beta_j + \gamma_{ik} + X_{ij}\theta + e_{ijk}
\]
where $\gamma_{ik}\sim N(0,\nu^2)$ and with each distinct participant in a cluster $i$ receiving their own index $k$. With the added individual-level random effects, an individual analysis using the GLMM or GEE approach would be the preferred approach as otherwise carrying out a WLS with the correct covariance structure would likely be difficult.
\\

Recent work in the literature have examined various properties of cohort designs in varying contexts. For example, \textcite{Kasza:2020} proposed sample size and power formulas for general open cohort longitudinal CRTs. \textcite{Li:2020} proposed sample size and power calculation procedures for cohort SW-CRTs with a decay correlation structure (e.g., when participants are expected to change over time or participants in each cluster are slowly replaced by new ones). The commentary by \textcite{Hooper:2019} has called for a more fundamental change in modeling SW-CRTs with continuous recruitment and proposed thinking about time as a continuous variable.

\subsection{Hierarchical designs}

The SW-CRT designs discussed so far in this report can all be considered as a hierarchical design with two levels: the clusters and the participants within a cluster. It is possible that the data collection procedure in a trial may lead to more than two levels of clustering. For example, the CHANGE trial \parencite{Lescure:2021} that investigated the effect of education and awareness activities on hand hygiene in nurses involved four levels of clustering. Data were collected on hand hygiene opportunities (level 1) for the nurses (level 2) in certain wards (level 3) of several nursing homes (level 4).
\\

Following the idea of the basic model where random effects are used to model the correlation between the units within a cluster, the correlation due to higher-level clustering can also be incorporated in the model by introducing additional random effects. For example, if we have a three-level trial measuring patients (level 1) in a ward (level 2) in a hospital (level 3) and the visited ward is different at every time point, then a suitable model for this trial may be
\[
Y_{ijk} = \mu + \alpha_i + \beta_j + \gamma_{ij} + X_{ij}\theta + e_{ijk}
\]
where $\gamma_{ij}\sim N(0,\nu^2)$ (i.e., a cluster-time random effect). Depending on the complexity of the hierarchy, cluster-level analyses may be possible but otherwise an individual analysis using GLMM or GEE would be preferred.
\\

\textcite{Teerenstra:2019} showed that the general impact of the multi-level structure is the inflation of the cluster mean variances at each level. They also proposed sample size and power calculation formulas for SW-CRTs with more than two levels, which were recently further developed in the work by \textcite{DavisPlourde:2021}.

\subsection{Bayesian approaches}

While the approaches for analyzing SW-CRTs have been predominantly frequentist, a number of Bayesian perspectives have appeared within the last few years. The Bayesian approach is a possible alternative to the standard approach for most SW-CRT designs, with the main difference being that priors are placed on what would be the fixed effects and hyperparameters of the basic model. Fitting the Bayesian model would also require a different approach from the frequentist one. Gibbs sampling, a Markov chain Monte Carlo method, is commonly used to fit the model in the Bayesian SW-CRT literature \parencite{Cunanan:2016,Zhan:2021}.
\\

The British Columbia Telehealth Trial (BCTT) is an example of one trial that has been analyzed through a Bayesian approach \parencite{Cunanan:2016}. The BCTT trial was a cross-sectional study that evaluated the effect of telehealth support on improving patient compliance with recommended disease-specific action plans. \citeauthor{Cunanan:2016} fitted a model using largely uninformative priors on the parameters and computed the power and size through a simulation study. It is noted that one of the strengths of Bayesian approaches over frequentist ones is the access to a full posterior distribution for inference as opposed to relying on point estimates and asymptotic results.
\\

The more recent work by \textcite{Zhan:2021} explored how informative priors on the time effects in Bayesian models may avoid the overestimation of a required sample size in the design of SW-CRTs. Their results suggest that sample size calculations are consistent with that of frequentist calculations when uninformed priors are used, and that well-specified informative priors reduce the needed sample size. When the priors are misspecified, the introduced bias stays relatively small unless the prior is greatly misspecified. The ability to specify a prior is both a strength and a common point of criticism of Bayesian methods: informative priors may improve the efficiency of estimators relative to that of frequentist approaches, but the choice of a prior is subjective and misspecified priors may instead result in worse efficiency.


\section{Conclusion} \label{sec:discussion}


\newpage


\printbibliography


\newpage


\appendix
\section{Appendix} \label{apx}

This appendix section includes missing technical details from the paper by \textcite{Hussey:2007} that we exclude from our summary in Section~\ref{sec:summary} for the sake of brevity.

\subsection{Intuition behind ICC and CV} \label{apx:correlation}

The intraclass correlation (ICC) and coefficient of variation (CV) are two quantities commonly used to characterize the effect of within-cluster correlation in CRTs. The ICC is given by
\[
\rho = \frac{\tau^2}{\tau^2+\sigma^2}
\]
and can be viewed as comparing the cluster variation (the numerator) to the individual variation (the denominator). The ICC is bounded between 0 and 1 where it is 0 if and only if $\tau^2=0$ (there is no variation between clusters) and $1$ if and only if $\sigma^2=0$ (there is no variation between individuals). A quantity related to the ICC is the variance inflation factor $\nu=1+(N-1)\rho\geq1$, which gets its name from rewriting the cluster mean variance as
\begin{align*}
\Var(\bar{Y}_{ij.}) &= \frac{\sigma^2}{N} + \tau^2 \\
&= \frac{\tau^2+\sigma^2}{N} + \tau^2 - \frac{\tau^2}{N} \\
&= \frac{\tau^2+\sigma^2}{N} + \left(\frac{\tau^2(N-1)}{N}\right)\left(\frac{\tau^2+\sigma^2}{\tau^2+\sigma^2}\right) \\
&= \left(\frac{\tau^2+\sigma^2}{N}\right)\left(1+\frac{(N-1)\tau^2}{\tau^2+\sigma^2}\right) \\
&= \left(\frac{\tau^2+\sigma^2}{N}\right)\nu \;.
\end{align*}
When the individuals are all independent and there is no variation between clusters ($\tau^2=0$), we have $\rho=0$, $\nu=1$, and $\Var(\bar{Y}_{ij.})=\frac{\sigma^2}{N}$. Hence, any amount of cluster variation $\tau^2>0$ then ``inflates'' the mean variance relative to the independent case.
\\

The CV is given by $c=\frac{\tau}{\mu}$ and can be viewed as the cluster standard deviation relative to the mean. The CV is unitless and therefore may be useful when comparing the degree of cluster variation in one trial to another without needing to account for scale.

\subsection{Within-cluster estimator for treatment effect} \label{apx:estimator}

Let $t_i\in\{1,\ldots,T-1\}$ be the last time point at which cluster $i$ receives the control/existing intervention. When there are no time effects on the outcome, the within-cluster estimator for the treatment effect given by \textcite{Hussey:2007} is
\[
\tilde{\theta} = \frac{1}{I}\sum_{i=1}^I\left(\frac{\sum_{j=t_i+1}^T\bar{Y}_{ij.}}{T-t_i} - \frac{\sum_{j=1}^{t_i}\bar{Y}_{ij.}}{t_i}\right)\;.
\]
Each term in the sum corresponds to a particular cluster, and each term calculates the difference in the mean outcome between the intervention and the control periods of that cluster. Note that the estimator is only unbiased when there are no time effects, as
\begin{align*}
\E[\tilde{\theta}] &= \frac{1}{I}\sum_{i=1}^I\left(\frac{\sum_{j=t_i+1}^T\E[\bar{Y}_{ij.}]}{T-t_i} - \frac{\sum_{j=1}^{t_i}\E[\bar{Y}_{ij.}]}{t_i}\right) \\
&= \frac{1}{I}\sum_{i=1}^I\left(\frac{\sum_{j=t_i+1}^T(\mu+\alpha_i+\beta_j+\theta)}{T-t_i} - \frac{\sum_{j=1}^{t_i}(\mu+\alpha_i+\beta_j)}{t_i}\right) \\
&= \frac{1}{I}\sum_{i=1}^I\left(\frac{\sum_{j=t_i+1}^T(\mu+\alpha_i+\theta)}{T-t_i} - \frac{\sum_{j=1}^{t_i}(\mu+\alpha_i)}{t_i}\right) \tag{*} \\
&= \frac{1}{I}\sum_{i=1}^I(\mu+\alpha_i+\theta-\mu-\alpha_i) \\
&= \theta
\end{align*}
where the line $(*)$ follows from the assumption of no time effects. When there are time effects, the estimator is biased with the bias being
\begin{align*}
\bias(\tilde{\theta},\theta) &= \E[\tilde{\theta}] - \theta \\
&= \frac{1}{I}\sum_{i=1}^I\left(\frac{\sum_{j=t_i+1}^T(\mu+\alpha_i+\beta_j+\theta)}{T-t_i} - \frac{\sum_{j=1}^{t_i}(\mu+\alpha_i+\beta_j)}{t_i}\right) - \theta \\
&= \frac{1}{I}\sum_{i=1}^I\left(\frac{\sum_{j=t_i+1}^T\beta_j}{T-t_i} + \theta - \frac{\sum_{j=1}^{t_i}\beta_j}{t_i}\right) - \theta \\
&= \frac{1}{I}\sum_{i=1}^I\left(\frac{\sum_{j=1}^T\beta_jX_{ij}}{T-t_i} - \frac{\sum_{j=1}^T\beta_j(1-X_{ij})}{t_i}\right) \\
&= \frac{1}{I}\sum_{j=1}^T\beta_j\sum_{i=1}^I\frac{t_iX_{ij} - (T-t_i)(1-X_{ij})}{t_i(T-t_i)} \\
&= \sum_{j=1}^T\beta_j\sum_{i=1}^I\frac{t_i - T(1-X_{ij})}{It_i(T-t_i)}
\end{align*}
with the form in the last line being the one given by \citeauthor{Hussey:2007}.


\subsection{Relative efficiency of WLS and within-cluster estimator} \label{apx:efficiency}

The relative efficiency of the WLS estimator $\hat{\theta}$ versus the within-cluster estimator $\tilde{\theta}$ is given by the inverse ratio of their variances, i.e.,
\[
\mathrm{efficiency}(\hat{\theta},\tilde{\theta}) = \frac{\Var(\tilde{\theta})}{\Var(\hat{\theta})} \;.
\]
The WLS estimator is more efficient than the within-cluster estimator if the ratio is greater than 1, and vice versa if the ratio is less than 1. \textcite{Hussey:2007} state that when there are no time effects, the ratio is
\[
\mathrm{efficiency}(\hat{\theta},\tilde{\theta}) = \frac{\sum_{i=1}^I\left(\frac{1}{t_i}+\frac{1}{T-t_i}\right)\left((ITU-U^2)\frac{\sigma^2}{N}+IT(TU-V)\tau^2\right)}{I^3\left(\frac{\sigma^2}{N}+T\tau^2\right)} \;.
\]
\textcite{Liao:2015} have shown that there is a missing factor in the denominator, and that the correct ratio is
\[
\mathrm{efficiency}(\hat{\theta},\tilde{\theta}) = \frac{\sum_{i=1}^I\left(\frac{1}{t_i}+\frac{1}{T-t_i}\right)\left((ITU-U^2)\frac{\sigma^2}{N}+IT(TU-V)\tau^2\right)}{I^3T\left(\frac{\sigma^2}{N}+T\tau^2\right)} \;.
\]
We show how this quantity is obtained in the following subsections.

\iffalse
Note: $\Var(\hat{\theta})$ is not the one given above as it is assumed there are no time effects.
\begin{align*}
\mathrm{efficacy}(\hat{\theta},\tilde{\theta}) &= \frac{\Var(\tilde{\theta})}{\Var(\hat{\theta})} \\
&= \frac{\left(\frac{\sigma^2}{NI^2}\sum_i\left(\frac{1}{t_i}+\frac{1}{T-t_i}\right)\right)}{\left(\frac{I\sigma^2\left(\frac{\sigma^2}{N}+T\tau^2\right)}{(IU-W)\sigma^2+N(U^2+ITU-TW-IV)\tau^2}\right)} \\
&= \frac{\sum_i\left(\frac{1}{t_i}+\frac{1}{T-t_i}\right)\left((IU-W)\sigma^2+N(U^2+ITU-TW-IV)\tau^2\right)}{NI^3\left(\frac{\sigma^2}{N}+T\tau^2\right)} \\
&= \frac{\sum_i\left(\frac{1}{t_i}+\frac{1}{T-t_i}\right)\left((IU-W)\frac{\sigma^2}{N}+(U^2+ITU-TW-IV)\tau^2\right)}{I^3\left(\frac{\sigma^2}{N}+T\tau^2\right)} 
\end{align*}
\fi

\subsubsection{Variance of within-cluster estimator} \label{apx:withincluster}

The variance of the within-cluster estimator $\tilde{\theta}$ of $\theta$ is given by
\begin{align*}
\Var(\tilde{\theta}) &= \frac{1}{I^2}\sum_{i=1}^I\Var\left(\frac{\sum_{j=t_i+1}^T\bar{Y}_{ij.}}{T-t_i} - \frac{\sum_{j=1}^{t_i}\bar{Y}_{ij.}}{t_i}\right) \\
&= \frac{1}{I^2}\sum_{i=1}^I\Var\left(\frac{\sum_{j=t_i+1}^T\left(\beta_j+\theta+\frac{1}{N}\sum_{k=1}^Ne_{ijk}\right)}{T-t_i} - \frac{\sum_{j=1}^{t_i}\left(\beta_j+\frac{1}{N}\sum_{k=1}^Ne_{ijk}\right)}{t_i}\right) \tag{*} \\
&= \frac{\sigma^2}{NI^2}\sum_{i=1}^I\left(\frac{1}{T-t_i} + \frac{1}{t_i}\right)
\end{align*}
where the line $(*)$ follows because $\mu$ and $\alpha_i$ cancel out between the two terms.

\subsubsection{Variance of WLS estimator}

Let $\hat{\theta}$ denote the WLS estimator of $\theta$ extracted from the WLS solution $\hat{\eta}=(\hat{\mu},\hat{\beta}_1,\ldots,\hat{\beta}_{T-1},\hat{\theta})$. Under the assumption that there are no time effects, the WLS solution has the form $\hat{\eta} = (\hat{\mu},\hat{\theta})$ and is obtained from
\[
\hat{\eta} = (\bfX^\T\bfV^{-1}\bfX)^{-1}\bfX^\T\bfV^{-1}\bfy
\]
where $\bfX$ is the $IT\times 2$ design matrix given by
\[
\bfX =
\begin{bmatrix}
1 & X_{11} \\
\vdots & \vdots \\
1 & X_{IT}
\end{bmatrix}\;,
\]
$\bfy$ is the vector of cluster means of length $IT$, and $\bfV=\Var(\bfy)$ is the $IT\times IT$ block diagonal matrix with each $T\times T$ block $\bfV_i$, $i\in\{1,\ldots,I\}$, describing the correlation structure of a cluster over time given by
\[
\bfV_i =
\begin{bmatrix}
\tau^2 + \frac{\sigma^2}{N} & \tau^2 & \ldots & \tau^2 \\
\tau^2 & \ddots & & \vdots \\
\vdots & & \ddots & \tau^2 \\
\tau^2 & \ldots & \tau^2 & \tau^2 + \frac{\sigma^2}{N}
\end{bmatrix}\;.
\]
Let $\bfe_i$ represent the unit column vector with a 1 in the $i$-th position. The variance of $\hat{\theta}$ is given by
\begin{align*}
\Var(\hat{\theta}) &= \bfe_2^\T(\bfX^\T\bfV^{-1}\bfX)^{-1}\bfX^\T\bfV^{-1}\Var(\bfy)\bfV^{-1}\bfX^\T(\bfX^\T\bfV^{-1}\bfX)^{-1}\bfe_2 \\
&= \bfe_2^\T(\bfX^\T\bfV^{-1}\bfX)^{-1}\bfe_2
\end{align*}
A closed-form for the variance of $\hat{\theta}$ is possible for the LMM when $X_{ij}\in\{0,1\}$ \parencite{Hussey:2007}. We follow the derivation by \textcite{Liao:2015} for the closed-form while clarifying their steps in the process.
\\

Note that we can rewrite $\bfV_i=\frac{\sigma^2}{N}\bfI_T+\tau^2\onevec_T\onevec_T^\T$. By the Sherman-Morrison formula, we have
\begin{align*}
\bfV_i^{-1} &= \frac{N}{\sigma^2}\bfI_T - \frac{\frac{N^2\tau^2}{\sigma^4}}{1+\frac{N\tau^2}{\sigma^2}\onevec_T^\T\bfI_T\onevec_T}\bfI_T\onevec_T\onevec_T^\T\bfI_T \\
&= \frac{N}{\sigma^2}\left(\bfI_T - \frac{N\tau^2}{\sigma^2+NT\tau^2}\onevec_T\onevec_T^\T\right) \\
&= \frac{N^2}{\sigma^4+NT\sigma^2\tau^2}
\begin{bmatrix}
\frac{\sigma^2}{N}+(T-1)\tau^2 & -\tau^2 & \ldots & -\tau^2 \\
-\tau^2 & \ddots & & \vdots \\
\vdots & & \ddots & -\tau^2 \\
-\tau^2 & \ldots & -\tau^2 & \frac{\sigma^2}{N}+(T-1)\tau^2
\end{bmatrix}
\end{align*}
We then have
\begin{align*}
\bfV^{-1}\bfX &= \frac{N^2}{\sigma^4+NT\sigma^2\tau^2}
\begin{bmatrix}
\frac{\sigma^2}{N} & \left(\frac{\sigma^2}{N}+(T-1)\tau^2\right)X_{11} -\tau^2\sum_{j=2}^TX_{1j} \\
\vdots & \vdots \\
\frac{\sigma^2}{N} & \left(\frac{\sigma^2}{N}+(T-1)\tau^2\right)X_{IT} -\tau^2\sum_{j=1}^{T-1}X_{Ij}
\end{bmatrix} \;, \\
%= \frac{N}{\sigma^2+NT\tau^2}
%\begin{bmatrix}
%1 & \sum_{j=1}^TX_{1j} \\
%\vdots & \vdots \\
%1 & \sum_{j=1}^TX_{Ij}
%\end{bmatrix} \;, \\
\bfX^\T\bfV^{-1}\bfX &= \frac{N^2}{\sigma^4+NT\sigma^2\tau^2}
\begin{bmatrix}
\frac{\sigma^2}{N}IT & \frac{\sigma^2}{N}U \\
\frac{\sigma^2}{N}U & W
\end{bmatrix}
\end{align*}
with
\begin{align*}
U & = \sum_{i=1}^I\sum_{j=1}^TX_{ij} \;, \\
W &= \sum_{i=1}^I\sum_{j=1}^T\left(\left(\frac{\sigma^2}{N}+(T-1)\tau^2\right)X_{ij}^2 -\tau^2X_{ij}\left(\sum_{k=1}^TX_{ik}-X_{ij}\right)\right) \\
&= \sum_{i=1}^I\sum_{j=1}^T\left(\left(\frac{\sigma^2}{N}+T\tau^2\right)X_{ij}^2 -\tau^2X_{ij}\sum_{k=1}^TX_{ik}\right) \\
%&= \frac{\sigma^2}{N}\sum_{i=1}^I\sum_{j=1}^TX_{ij}^2+\tau^2\sum_{i=1}^I\left(T\sum_{j=1}^TX_{ij}^2 - \sum_{j=1}^TX_{ij}\sum_{k=1}^TX_{ik}\right) \\
%&= \frac{\sigma^2}{N}\sum_{i=1}^I\sum_{j=1}^TX_{ij}^2+\tau^2\sum_{i=1}^I\sum_{j=1}^T\left(TX_{ij}^2 - X_{ij}\sum_{k=1}^TX_{ik}\right) \\
%&= \frac{\sigma^2}{N}\sum_{i=1}^I\sum_{j=1}^TX_{ij}^2+\tau^2\sum_{i=1}^I\left(\sum_{j=1}^TX_{ij}\left(TX_{ij} - \sum_{k=1}^TX_{ik}\right)\right) \\
%&= \frac{\sigma^2}{N}\sum_{i=1}^I\sum_{j=1}^TX_{ij}^2+\tau^2\sum_{i=1}^I\left(\sum_{j=1}^TX_{ij}\left((T-1)X_{ij} - \sum_{k\neq j}^TX_{ik}\right)\right) \\
%&= \frac{\sigma^2}{N}\sum_{i=1}^I\sum_{j=1}^TX_{ij}^2+\tau^2\sum_{i=1}^I\sum_{j=1}^T\left((T-1)X_{ij}^2 - \sum_{k\neq j}^TX_{ij}X_{ik}\right) \\
%&= \frac{\sigma^2}{N}\sum_{i=1}^I\sum_{j=1}^TX_{ij}^2+\tau^2\sum_{i=1}^I\left((T-1)\sum_{j=1}^TX_{ij}^2 - \sum_{j=1}^T\sum_{k\neq j}^TX_{ij}X_{ik}\right) \\
&= \left(\frac{\sigma^2+NT\tau^2}{N}\right)\sum_{i=1}^I\sum_{j=1}^TX_{ij}^2 -\tau^2\sum_{i=1}^I\sum_{j=1}^T\left(X_{ij}^2+\sum_{k\neq j}^TX_{ij}X_{ik}\right) \\
&= \left(\frac{\sigma^2+NT\tau^2}{N}\right)U -\tau^2\sum_{i=1}^I\left(\sum_{j=1}^TX_{ij}\right)^2
\end{align*}
where $U$ in the last line follows because $X_{ij}=X_{ij}^2\in\{0,1\}$. Let $V = \sum_{i=1}^I\left(\sum_{j=1}^TX_{ij}\right)^2$. Therefore,
\[
\bfX^\T\bfV^{-1}\bfX =
\begin{bmatrix}
\frac{NIT}{\sigma^2+NT\tau^2} & \frac{NU}{\sigma^2+NT\tau^2} \\
\frac{NU}{\sigma^2+NT\tau^2} & \frac{NU}{\sigma^2} - \frac{N^2\tau^2V}{\sigma^4+NT\sigma^2\tau^2}
\end{bmatrix} \;.
\]
By the formula for the inverse of a $2\times 2$ matrix, we then have
\begin{align*}
\Var(\hat{\theta}) &= \bfe_2^\T(\bfX^\T\bfV^{-1}\bfX)^{-1}\bfe_2 \\
&= \left(\frac{NIT}{\sigma^2+NT\tau^2}\right)\left(\left(\frac{NIT}{\sigma^2+NT\tau^2}\right)\left(\frac{NU}{\sigma^2} - \frac{N^2\tau^2V}{\sigma^4+NT\sigma^2\tau^2}\right)-\left(\frac{NU}{\sigma^2+NT\tau^2}\right)^2\right)^{-1} \\
%&= \left(\frac{NIT}{\sigma^2+NT\tau^2}\right)\left(\left(\frac{NIT}{\sigma^2+NT\tau^2}\right)\left(\frac{NU\sigma^2+N^2\tau^2(UT-V)}{\sigma^4+NT\sigma^2\tau^2}\right)-\left(\frac{NU}{\sigma^2+NT\tau^2}\right)^2\right)^{-1} \\
&= \left(\frac{NU\sigma^2+N^2\tau^2(UT-V)}{\sigma^4+NT\sigma^2\tau^2}-\frac{NU^2}{IT(\sigma^2+NT\tau^2)}\right)^{-1} \\
%&= \frac{IT\sigma^2(\sigma^2+NT\tau^2)}{NUIT\sigma^2+N^2IT(UT-V)\tau^2-NU^2\sigma^2} \\
&= \frac{IT\frac{\sigma^2}{N}\left(\frac{\sigma^2}{N}+T\tau^2\right)}{(ITU-U^2)\frac{\sigma^2}{N}+IT(UT-V)\tau^2} \;.
\end{align*}

\subsubsection{Relative efficiency}

From the previous sections, the relative efficiency of the WLS and the within-cluster estimator is then directly obtained from the ratio
\begin{align*}
\mathrm{efficiency}(\hat{\theta},\tilde{\theta}) &= \frac{\Var(\tilde{\theta})}{\Var(\hat{\theta})} \\
&= \frac{\sigma^2}{NI^2}\sum_{i=1}^I\left(\frac{1}{T-t_i} + \frac{1}{t_i}\right)\left(\frac{IT\frac{\sigma^2}{N}\left(\frac{\sigma^2}{N}+T\tau^2\right)}{(ITU-U^2)\frac{\sigma^2}{N}+IT(UT-V)\tau^2}\right)^{-1} \\
&= \frac{\sum_{i=1}^I\left(\frac{1}{T-t_i} + \frac{1}{t_i}\right)\left((ITU-U^2)\frac{\sigma^2}{N}+IT(UT-V)\tau^2\right)}{I^3T\left(\frac{\sigma^2}{N}+T\tau^2\right)} \;.
\end{align*}

\subsubsection{Time effect on efficiency}

We reconstruct the proof for the efficiency of the WLS and within-cluster estimators under the assumption of no time effects in this section.

\begin{proposition}[\textcite{Liao:2015} with minor errors corrected] \label{prop:efficiency}
Under the model described in Section~\ref{sec:model} and under the assumption that there are no time effects, we have
\[
\Var(\hat{\theta}) \leq \Var(\tilde{\theta}) \;.
\]
\end{proposition}
\begin{proof}
Note that we can rewrite the quantities
\begin{align*}
U &= \sum_{i=1}^I\sum_{j=1}^TX_{ij} = \sum_{i=1}^I(T-t_i) = IT - \sum_{i=1}^It_i \;, \\
V &= \sum_{i=1}^I\left(\sum_{j=1}^TX_{ij}\right)^2 = \sum_{i=1}^I(T-t_i)^2 = IT^2 - 2T\sum_{i=1}^It_i + \sum_{i=1}^It_i^2 \;.
\end{align*}
Hence, we have
\begin{align*}
\mathrm{efficiency}(\hat{\theta},\tilde{\theta}) &= \frac{\sum_{i=1}^I\left(\frac{1}{T-t_i} + \frac{1}{t_i}\right)\left((ITU-U^2)\frac{\sigma^2}{N}+IT(UT-V)\tau^2\right)}{I^3T\left(\frac{\sigma^2}{N}+T\tau^2\right)} \\
&= T\sum_{i=1}^I\frac{1}{t_i(T-t_i)}\left(\frac{(U-\frac{U^2}{IT})\frac{\sigma^2}{N}+(UT-V)\tau^2}{I^2\left(\frac{\sigma^2}{N}+T\tau^2\right)}\right) \\
&= T\left(\sum_{i=1}^I\frac{1}{t_i(T-t_i)}\right)\left(\frac{\left(\sum_{i=1}^It_i-\frac{\left(\sum_{i=1}^It_i\right)^2}{IT}\right)\frac{\sigma^2}{N}+(UT-V)\tau^2}{I^2\left(\frac{\sigma^2}{N}+T\tau^2\right)}\right) \\
&= T\left(\sum_{i=1}^I\frac{1}{t_i(T-t_i)}\right)\left(\frac{\left(\sum_{i=1}^It_i-\frac{\left(\sum_{i=1}^It_i\right)^2}{IT}\right)\frac{\sigma^2}{N}+\left(\sum_{i=1}^It_i-\frac{\sum_{i=1}^It_i^2}{T}\right)T\tau^2}{I^2\left(\frac{\sigma^2}{N}+T\tau^2\right)}\right) \;.
\end{align*}
Let $\bfe_I$ be a vector of ones and $\bft=(t_1,\ldots,t_I)$. By the Cauchy-Schwarz inequality,
\[
\sum_{i=1}^It_i^2 = \|\bft\|^2 \geq \frac{|\bfe_I^\T\bft|^2}{\|\bfe_I\|^2} = \frac{\left(\sum_{i=1}^It_i\right)^2}{I} \;.
\]
Then
\begin{align*}
\mathrm{efficiency}(\hat{\theta},\tilde{\theta}) &\geq T\left(\sum_{i=1}^I\frac{1}{t_i(T-t_i)}\right)\left(\frac{\left(\sum_{i=1}^It_i-\frac{\sum_{i=1}^It_i^2}{T}\right)\frac{\sigma^2}{N}+\left(\sum_{i=1}^It_i-\frac{\sum_{i=1}^It_i^2}{T}\right)T\tau^2}{I^2\left(\frac{\sigma^2}{N}+T\tau^2\right)}\right) \\
&= \left(\sum_{i=1}^I\frac{1}{t_i(T-t_i)}\right)\left(\frac{T\sum_{i=1}^It_i-\sum_{i=1}^It_i^2}{I^2}\right) \\
&= \frac{1}{I^2}\left(\sum_{i=1}^I\frac{1}{t_i(T-t_i)}\right)\left(\sum_{i=1}^It_i(T-t_i)\right) \;.
\end{align*}
Again, by the Cauchy-Schwarz inequality,
\[
\left(\sum_{i=1}^I\frac{1}{t_i(T-t_i)}\right)^{\frac{1}{2}}\left(\sum_{i=1}^It_i(T-t_i)\right)^\frac{1}{2} \geq \left|\sum_{i=1}^I\sqrt{\frac{t_i(T-t_i)}{t_i(T-t_i)}}\right| = I \;.
\]
Thus,
\[
\mathrm{efficiency}(\hat{\theta},\tilde{\theta}) = \frac{\Var(\tilde{\theta})}{\Var(\hat{\theta})} \geq 1 \;.
\]
\end{proof}

Note that even in the case $\tau^2=0$, the above result still appears to hold unlike what \textcite{Hussey:2007} claim.

\begin{corollary}
Under the model described in Section~\ref{sec:model} and under the assumptions that there are no time effects and that $\tau^2=0$, we have
\[
\Var(\hat{\theta}) \leq \Var(\tilde{\theta}) \;.
\]
\end{corollary}
\begin{proof}
The efficiency when $\tau^2=0$ is
\[
\mathrm{efficiency}(\hat{\theta},\tilde{\theta}) = \left(\sum_{i=1}^I\frac{1}{t_i(T-t_i)}\right)\left(\frac{T\sum_{i=1}^It_i-\frac{\left(\sum_{i=1}^It_i\right)^2}{I}}{I^2}\right) \;.
\]
Then by the same argument as in Proposition~\ref{prop:efficiency},
\[
\mathrm{efficiency}(\hat{\theta},\tilde{\theta}) = \frac{\Var(\tilde{\theta})}{\Var(\hat{\theta})} \geq 1 \;.
\]
\end{proof}


\subsection{Wald test and power} \label{apx:wald}

\textcite{Hussey:2007} prescribe using a Wald test to obtain an approximate power for testing the hypothesis $H_0:\theta=0$ versus $H_a:\theta=\theta_a$. For some estimator $\hat{\theta}$ of $\theta$ that is normally-distributed (either exactly under assumptions or approximately based on large samples), the test statistic in the Wald test is
\[
Z = \frac{\hat{\theta}}{\sqrt{\Var(\hat{\theta})}}
\]
which has an (approximate) standard normal distribution under $H_0$. Under $H_a$, the statistic $Z$ has an (approximate) normal distribution with mean $\frac{\theta_a}{\sqrt{\Var(\hat{\theta})}}$ and variance 1. Let $Z_{1-\frac{\alpha}{2}}$ be the $\left(1-\frac{\alpha}{2}\right)$-th critical value of the standard normal distribution for significance level $\alpha$. The power of the two-tailed test is then
\begin{align*}
\P\left(\left.Z<-Z_{1-\frac{\alpha}{2}}\right|H_a\right) + \P\left(\left.Z>Z_{1-\frac{\alpha}{2}}\right|H_a\right) &= \P\left(\left.Z-\frac{\theta_a}{\sqrt{\Var(\hat{\theta})}}<-Z_{1-\frac{\alpha}{2}}-\frac{\theta_a}{\sqrt{\Var(\hat{\theta})}}\right|H_a\right) \\
&\quad + \P\left(\left.Z-\frac{\theta_a}{\sqrt{\Var(\hat{\theta})}}>Z_{1-\frac{\alpha}{2}}-\frac{\theta_a}{\sqrt{\Var(\hat{\theta})}}\right|H_a\right) \\
&= \Phi\left(-\frac{\theta_a}{\sqrt{\Var(\hat{\theta})}}-Z_{1-\frac{\alpha}{2}}\right) + 1-\Phi\left(Z_{1-\frac{\alpha}{2}}-\frac{\theta_a}{\sqrt{\Var(\hat{\theta})}}\right) \\
&= \Phi\left(-\frac{\theta_a}{\sqrt{\Var(\hat{\theta})}}-Z_{1-\frac{\alpha}{2}}\right) + \Phi\left(\frac{\theta_a}{\sqrt{\Var(\hat{\theta})}} - Z_{1-\frac{\alpha}{2}}\right)
\end{align*}
where $\Phi$ is the cumulative distribution function of the standard normal. Notice that if $\theta_a>0$ and is not too small, then the first term is approximately 0 and so the power is approximately 
\[
\P\left(\left.Z<-Z_{1-\frac{\alpha}{2}}\right|H_a\right) + \P\left(\left.Z>Z_{1-\frac{\alpha}{2}}\right|H_a\right) \approx \P\left(\left.Z>Z_{1-\frac{\alpha}{2}}\right|H_a\right) = \Phi\left(\frac{\theta_a}{\sqrt{\Var(\hat{\theta})}} - Z_{1-\frac{\alpha}{2}}\right) \;,
\]
which is the power given by \textcite{Hussey:2007}. The other term dominates when $\theta_a<0$ and is not too small (in absolute value). Note that this calculation is also approximate if $Z$ is only approximately normally distributed or if $\Var(\hat{\theta})$ needs to be estimated.

\subsection{Measured time points and delayed treatment effect on power} \label{apx:power}

It can be seen from the power calculation in Appendix~\ref{apx:wald} that the power depends on the variance of the estimator $\hat{\theta}$. For estimators that have a relatively large variance, the power decreases to the significance level where
\[
\Phi\left(-\frac{\theta_a}{\sqrt{\Var(\hat{\theta})}}-Z_{1-\frac{\alpha}{2}}\right) + \Phi\left(\frac{\theta_a}{\sqrt{\Var(\hat{\theta})}} - Z_{1-\frac{\alpha}{2}}\right) \approx 2\Phi\left(-Z_{1-\frac{\alpha}{2}}\right) = \alpha \;.
\]
For estimators that have a relatively small variance, one of the terms dominate and so a power greater than the significance level can be expected.
\\

From the above, it then follows that design factors and assumptions that affect the variance of the estimator will also affect the power of the study. \textcite{Hussey:2007} briefly discuss how the number of measured time points and delays in the treatment effect affect power. We aim to provide more insight on their discussion in the following sections.

\subsubsection{Number of time points}

\textcite{Hussey:2007} state that the optimal power is achieved when only one cluster crosses over at each time point. We use the within-cluster estimator $\tilde{\theta}$ to illustrate this point. From Appendix~\ref{apx:withincluster}, the variance of the estimator is
\[
\Var(\tilde{\theta}) = \frac{\sigma^2}{NI^2}\sum_{i=1}^I\left(\frac{1}{T-t_i} + \frac{1}{t_i}\right) \;.
\]
Suppose that all $I>2$ clusters are assigned an unique crossover time ($t_i=i$ without loss of generality) and that $T=I+1$. Suppose that in another trial, the clusters are measured over $T-1$ time points and the $I$-th cluster shares its crossover time with another cluster $j\in\{1,\ldots,I-1\}$. The time factor in the estimator variance for this other trial with fewer time points is then
\begin{align*}
\sum_{i=1}^{I-1}\left(\frac{1}{T-t_i-1} + \frac{1}{t_i}\right) + \frac{1}{T-t_j-1} + \frac{1}{t_j} &= \sum_{i=1}^{I-1}\left(\frac{1}{I-i} + \frac{1}{i}\right) + \frac{1}{I-j} + \frac{1}{j} \\
&> \sum_{i=1}^{I-1}\left(\frac{1}{I-i} + \frac{1}{i}\right) + \frac{1}{I} + \frac{1}{I} \\
&= \sum_{i=1}^I\left(\frac{1}{I-i+1}+\frac{1}{i}\right) \\
&= \sum_{i=1}^I\left(\frac{1}{T-t_i}+\frac{1}{t_i}\right) \;.
\end{align*}
Similar arguments can be made for trials with even fewer time points. Thus, keeping everything but the number of time points fixed, the variance of the estimator is smallest when each cluster crosses over at its own time point. The increase in variance due to a reduced number of time points leads to a decrease in power.

\subsubsection{Delayed treatment effect} \label{apx:delay}

\textcite{Hussey:2007} state that delays in the treatment effect reduce power. Delayed treatment effects can be modeled by allowing $X_{ij}$ to be in $[0,1]$. We again use a within-cluster estimator $\tilde{\theta}$ to illustrate this point. Note that the estimator given by \citeauthor{Hussey:2007} needs to be modified to account for the delay in the treatment effect. Suppose that the delays $X_{ij}$ are known and that $X_{ij}\in(0,1)$ for at least one cluster $i$ and time point $j$. Then an unbiased estimator (assuming that there are no separate time effects, i.e., $\beta_j=0$) is given by
\[
\tilde{\theta} = \left(\sum_{i=1}^I\sum_{j=1}^T\frac{X_{ij}}{T-t_i}\right)^{-1}\sum_{i=1}^I\left(\frac{\sum_{j=t_i+1}^T\bar{Y}_{ij.}}{T-t_i} - \frac{\sum_{j=1}^{t_i}\bar{Y}_{ij.}}{t_i}\right)\;.
\]
This estimator is unbiased as
\begin{align*}
\E[\tilde{\theta}] &= \left(\sum_{i=1}^I\sum_{j=1}^T\frac{X_{ij}}{T-t_i}\right)^{-1}\sum_{i=1}^I\left(\frac{\sum_{j=t_i+1}^T\E[\bar{Y}_{ij.}]}{T-t_i} - \frac{\sum_{j=1}^{t_i}\E[\bar{Y}_{ij.}]}{t_i}\right) \\
&= \left(\sum_{i=1}^I\sum_{j=1}^T\frac{X_{ij}}{T-t_i}\right)^{-1}\sum_{i=1}^I\left(\frac{\sum_{j=t_i+1}^T(\mu+\alpha_i+X_{ij}\theta)}{T-t_i} - \frac{\sum_{j=1}^{t_i}(\mu+\alpha_i)}{t_i}\right) \\
&= \theta\left(\sum_{i=1}^I\sum_{j=1}^T\frac{X_{ij}}{T-t_i}\right)^{-1}\sum_{i=1}^I\sum_{j=t_i+1}^T\frac{X_{ij}}{T-t_i} \\
&= \theta
\end{align*}
where the last line follows because $X_{ij}=0$ for $j\in\{1,\ldots,t_i\}$. It follows from the derivation in Appendix~\ref{apx:withincluster} that the variance of the estimator is
\[
\Var(\tilde{\theta}) = \frac{\sigma^2}{N}\left(\sum_{i=1}^I\sum_{j=1}^T\frac{X_{ij}}{T-t_i}\right)^{-2}\sum_{i=1}^I\left(\frac{1}{T-t_i} + \frac{1}{t_i}\right) \;.
\]
Note that by assumption,
\begin{align*}
\sum_{i=1}^I\sum_{j=1}^T\frac{X_{ij}}{T-t_i} &= \sum_{i=1}^I\sum_{j=t_i+1}^T\frac{X_{ij}}{T-t_i} < \sum_{i=1}^I\sum_{j=t_i+1}^T\frac{1}{T-t_i} = \sum_{i=1}^I1 = I.
\end{align*}
Thus,
\[
\Var(\tilde{\theta}) = \frac{\sigma^2}{N}\left(\sum_{i=1}^I\sum_{j=1}^T\frac{X_{ij}}{T-t_i}\right)^{-2}\sum_{i=1}^I\left(\frac{1}{T-t_i} + \frac{1}{t_i}\right) > \frac{\sigma^2}{NI^2}\sum_{i=1}^I\left(\frac{1}{T-t_i} + \frac{1}{t_i}\right)
\]
which is the variance of the within-cluster estimator for the case of no delays in treatment effects. The increase in variance due to the delay in treatment effect leads to a decrease in power.




\iffalse
\subsection{Introduction}

\begin{itemize}

\item
In cluster randomized trials (CRTs), randomization of interventions are done at the level of groups rather than individuals. This is most useful when intervention can only be administered at the community-level, when contamination may be an issue, or for other reasons. The inidividual units within a cluster are correlated.

\item
In parallel CRTs, clusters are all assigned an intervention at a single time point. If cluster sizes are equal, t-tests/ANOVA may be used to compare cluster-level mean responses. Clusters may also be matched for a paired setup. When cluster sizes vary, individual level analyses such as generalized estimating equations (GEE) or random effects models may be used.

\item
In crossover CRTs, each interventions is applied to a cluster at different time points (with a possible ``washout'' period in between). The order of the interventions is randomized for each cluster. Crossover designs are less commonly used in CRTs due to extending the period of the study. Crossover CRTs usually use paired t-tests to make within-cluster comparisons.

\item
In a stepped wedge CRT, clusters cross from one intervention to another at different time points (typically all starting from the control intervention and all ending with the treatment intervention). The time at which each cluster crosses over is randomized. The stepped wedge CRT is useful when there are limited resources for applying the intervention. The key features of the stepped wedge CRT is that the crossover is unidirectional and that the intervention is never removed once implemented. The unidirectionality does complicate the analysis as the treatment effect cannot be estimated from only within-cluster comparisons.

\end{itemize}

\subsection{Model}

\begin{itemize}

\item
For a design with $I$ clusters, $T$ time points, $N$ individuals sampled per cluster per time interval, define the cluster means
\[
\mu_{ij} = \mu + \alpha_i + \beta_j + \theta X_{ij}
\]
where
\begin{itemize}
\item
$\alpha_i\sim N(0,\tau^2)$ is a random effect for cluster $i\in \{1,\ldots,I\}$
\item
$\beta_j$ is a fixed effect for time interval $j\in \{1,\ldots,T-1\}$ (with $\beta_T=0$ for identifiability)
\item
$X_{ij}$ is the intervention indicator in cluster $i$ at time $j$ (with 1 denoting intervention)
\item
$\theta$ is the treatment effect.
\end{itemize}

\item
Individual level responses are modelled as
\[
Y_{ijk} = \mu_{ij} + \epsilon_{ijk}
\]
where $\epsilon_{ijk}\overset{iid}{\sim} N(0,\sigma^2)$. Cluster means are modelled as
\[
\bar{Y}_{ij.} = \mu_{ij} + \bar{\epsilon}_{ij.}
\]
where $\bar{\epsilon}_{ij.} = \frac{\sum_k\epsilon_{ijk}}{N}$. Assume that $\epsilon_{ijk}$ are independent of $\alpha_i$.

\item
The variance of the individual-level response is
\[
\Var(Y_{ijk}) = \tau^2 + \sigma^2
\]
The variance of the cluster-level response is
\[
\Var(\bar{Y}_{ij.}) = \tau^2 + \frac{\sigma^2}{N} = \left(\frac{\tau^2+\sigma^2}{N}\right)(1+(N-1)\rho)
\]
where $(1+(N-1)\rho$ is the ``variance inflation factor'' and $\rho=\frac{\tau^2}{\tau^2+\sigma^2}$ is the intraclass correlation. Note:

\item
Some characterize the cluster effect on the variance using the coefficient of variation (CV) $\frac{\tau}{\mu}$.

\item
If the individual level responses are binary, then the cluster-level response $\bar{Y}_{ij.}$ is a proportion and it is assumed $\sigma^2=\mu(1-\mu)$.

\item
For varying-sized clusters, replace $N$ with $N_{ij}$.

\end{itemize}

\subsection{Data analysis}

\subsubsection{$\tau^2$ and $\sigma^2$ known}

If the variances $\tau^2$ and $\sigma^2$ are known, then estimates of the fixed effects can be estimated used weighted least squares (WLS) at the cluster-level. Let $\bfX$ be the $IT\times(T+1)$ design matrix of cluster-time means corresponding to parameter vector $\eta=(\mu,\beta_1,\ldots,\beta_{T-1},\theta)$. Let $\bfW$ be a $IT\times IT$ block diagonal matrix where each $T\times T$ block of $\bfW$ describes the correlation structure between the repeated cluster means over time and has the structure
\[
\begin{bmatrix}
\tau^2 + \frac{\sigma^2}{N} & \tau^2 & \ldots & \tau^2 \\
\tau^2 & \ddots & & \vdots \\
\tau^2 & & \ddots & \tau^2 \\
\tau^2 & \ldots & \tau^2 & \tau^2 + \frac{\sigma^2}{N}
\end{bmatrix}
\]
Then the fixed effect estimates are given by
\[
\hat{\eta} = (\bfX^T\bfW^{-1}\bfX)^{-1}\bfX^T\bfW^{-1}\bfy
\]

\subsubsection{$\tau^2$ and $\sigma^2$ unknown}

\begin{itemize}

\item
When the response is continuous and normally distributed, an empirical Bayes approach at the cluster-level is possible to estimate the fixed effects in the LMM. This approach also works for non-normal individual-level data when the cluster sizes are approximately equal.

\item
If the responses are non-normal and the cluster sizes vary, then analysis at the individual-level using GLMM or GEE is preferred.

\end{itemize}



\subsection{Effect of number of steps}

Optimal power is achieved when a single cluster crosses over to the intervention at each time. The loss of power is primarily due to the loss of measurement times rather than due to the loss of randomization times. The loss of power is also relatively independent of the CV.


\subsection{Delayed treatment effect}

If effect of intervention only reaches full effect $\theta$ after some time, then power is reduced. The delay may be modelled by allowing $X_{ij}$ to be fractional (although the given $\Var(\hat{\theta})$ is not valid in this case). Power is increased by adding additional measurement periods at the end of the trial or by increasing the time intervals.

\subsection{Simulation results}

Compared power of test for LMM, GEE and GLMM for varying levels of relative risk. In equal cluster size case, LMM $>$ GEE $>$ GLMM. In unequal cluster size case, GEE $\approx$ GLMM $>$ LMM. When cluster sizes vary significantly, it is suggested to do individual-level analyses. A jackknife estimate of the variance is suggested to maintain the size of the test in GEE/GLMM analyses.

\fi



\end{document}