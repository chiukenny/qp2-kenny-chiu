% This file provides examples of some useful macros for typesetting
% dissertations.  None of the macros defined here are necessary beyond
% for the template documentation, so feel free to change, remove, and add
% your own definitions.
%
% We recommend that you define macros to separate the semantics
% of the things you write from how they are presented.  For example,
% you'll see definitions below for a macro \file{}: by using
% \file{} consistently in the text, we can change how filenames
% are typeset simply by changing the definition of \file{} in
% this file.

%% The following is a directive for TeXShop to indicate the main file
%%!TEX root = 210824_entromin.tex

\newcommand{\NA}{\textsc{n/a}}	% for "not applicable"
\newcommand{\eg}{e.g.,\ }	% proper form of examples (\eg a, b, c)
\newcommand{\ie}{i.e.,\ }	% proper form for that is (\ie a, b, c)
\newcommand{\etal}{\emph{et al}}

% Some useful macros for typesetting terms.
\newcommand{\file}[1]{\texttt{#1}}
\newcommand{\class}[1]{\texttt{#1}}
\newcommand{\latexpackage}[1]{\href{http://www.ctan.org/macros/latex/contrib/#1}{\texttt{#1}}}
\newcommand{\latexmiscpackage}[1]{\href{http://www.ctan.org/macros/latex/contrib/misc/#1.sty}{\texttt{#1}}}
\newcommand{\env}[1]{\texttt{#1}}
\newcommand{\BibTeX}{Bib\TeX}

% Define a command \doi{} to typeset a digital object identifier (DOI).
% Note: if the following definition raise an error, then you likely
% have an ancient version of url.sty.  Either find a more recent version
% (3.1 or later work fine) and simply copy it into this directory,  or
% comment out the following two lines and uncomment the third.
\DeclareUrlCommand\DOI{}
\newcommand{\doi}[1]{\href{http://dx.doi.org/#1}{\DOI{doi:#1}}}
%\newcommand{\doi}[1]{\href{http://dx.doi.org/#1}{doi:#1}}

% Useful macro to reference an online document with a hyperlink
% as well with the URL explicitly listed in a footnote
% #1: the URL
% #2: the anchoring text
\newcommand{\webref}[2]{\href{#1}{#2}\footnote{\url{#1}}}

% epigraph is a nice environment for typesetting quotations
\makeatletter
\newenvironment{epigraph}{%
	\begin{flushright}
	\begin{minipage}{\columnwidth-0.75in}
	\begin{flushright}
	\@ifundefined{singlespacing}{}{\singlespacing}%
    }{
	\end{flushright}
	\end{minipage}
	\end{flushright}}
\makeatother

% \FIXME{} is a useful macro for noting things needing to be changed.
% The following definition will also output a warning to the console
\newcommand{\FIXME}[1]{\typeout{**FIXME** #1}\textbf{[FIXME: #1]}}

% Theorems
\declaretheoremstyle[%
  spaceabove=\topsep,%
  spacebelow=\topsep,%
  headfont=\bfseries,%
  headpunct={.},%
  postheadspace=.5em,%
  bodyfont=\itshape,%
  headformat=\NAME~\NUMBER%
]{namelessTheorem}
\declaretheorem[style=namelessTheorem,name=Theorem]{btheorem}
\declaretheorem[style=namelessTheorem,sibling=theorem,name=Proposition]{bproposition}
\declaretheorem[style=namelessTheorem,sibling=theorem,name=Lemma]{blemma}
\declaretheorem[style=namelessTheorem,sibling=theorem,name=Corollary]{bcorollary}

\declaretheoremstyle[%
  spaceabove=\topsep,%
  spacebelow=\topsep,%
  headfont=\bfseries,%
  headpunct={},%
  postheadspace=.5em,%
  bodyfont=\itshape,%
  headformat=\NAME~\NUMBER%
]{namelessNoDotTheorem}
\declaretheorem[style=namelessNoDotTheorem,sibling=theorem,name=Proposition]{bpropositionnd}

% Definitions
\declaretheoremstyle[%
  spaceabove=\topsep,%
  spacebelow=\topsep,%
  headfont=\bfseries,%
  headpunct={.},%
  postheadspace=.5em,%
  headformat=\NAME~\NUMBER,%
  qed=$\lrcorner$
]{namelessDefinition}
\declaretheorem[style=namelessDefinition,name=Definition]{bdefinition}

% Assumptions
%\theoremstyle{definition}
\declaretheoremstyle[%
  spaceabove=\topsep,%
  spacebelow=\topsep,%
  headfont=\bfseries,%
  headpunct={.},%
  postheadspace=.5em
]{assumptionBlock}
\declaretheorem[style=assumptionBlock,name=Assumption]{assumption}
%\newtheorem{assumption}{Assumption}

\newcommand{\botsep}{\vspace{4pt}}


% Math commands
\newcommand{\Ebb}{\mathbb{E}\xspace}
\newcommand{\Nbb}{\mathbb{N}\xspace}
\newcommand{\Pbb}{\mathbb{P}\xspace}
\newcommand{\Rbb}{\mathbb{R}\xspace}

\newcommand{\Bcal}{\mathcal{B}\xspace}
\newcommand{\Ocal}{\mathcal{O}\xspace}
\newcommand{\Pcal}{\mathcal{P}\xspace}
\newcommand{\Qcal}{\mathcal{Q}\xspace}
\newcommand{\Rcal}{\mathcal{R}\xspace}
\newcommand{\Scal}{\mathcal{S}\xspace}
\newcommand{\Ucal}{\mathcal{U}\xspace}
\newcommand{\Vcal}{\mathcal{V}\xspace}
\newcommand{\Xcal}{\mathcal{X}\xspace}

\newcommand{\Ascr}{\mathscr{A}\xspace}
\newcommand{\Bscr}{\mathscr{B}\xspace}
\newcommand{\Dscr}{\mathscr{D}\xspace}
\newcommand{\Fscr}{\mathscr{F}\xspace}
\newcommand{\Lscr}{\mathscr{L}\xspace}
\newcommand{\Mscr}{\mathscr{M}\xspace}
\newcommand{\Oscr}{\mathscr{O}\xspace}
\newcommand{\Vscr}{\mathscr{V}\xspace}
\newcommand{\Xscr}{\mathscr{X}\xspace}
\newcommand{\Yscr}{\mathscr{Y}\xspace}
\newcommand{\Zscr}{\mathscr{Z}\xspace}

\DeclareMathOperator*{\argmax}{arg\,max\;}
\DeclareMathOperator*{\argmin}{arg\,min\;}
\DeclareMathOperator*{\deq}{\;\overset{\text{d}}{=}\;}
\DeclareMathOperator*{\aseq}{\;\overset{\text{a.s.}}{=}\;}

\newcommand{\ballset}{\Bcal}
\newcommand{\orthset}{\Ocal}
\newcommand{\permset}{\Pcal}
\newcommand{\doubstocset}{\Qcal}

\newcommand{\bigO}{\Oscr}

\newcommand{\ones}{\mathbf{1}}
\newcommand{\hatmu}{\widehat{\mu}}
\newcommand{\hatB}{\widehat{B}}
\newcommand{\hatD}{\widehat{D}}
\newcommand{\hatU}{\widehat{U}}
\newcommand{\hatV}{\widehat{V}}
\newcommand{\hatY}{\widehat{Y}}
\newcommand{\hatZ}{\widehat{Z}}
\newcommand{\tildeA}{\widetilde{A}}
\newcommand{\tildeR}{\widetilde{R}}
\newcommand{\tildemu}{\widetilde{\mu}}

\newcommand{\todo}{\textcolor{red}{TODO}}

% Names
\newcommand{\R}{\textsc{R}\xspace}
\newcommand{\EntrTwo}{Entromin$_2$\xspace}

\newcommand{\mre}{\textsc{MRE}\xspace}
\newcommand{\Var}{\mathrm{Var}}
\newcommand{\bias}{\mathrm{bias}}
\newcommand{\barY}{\bar{Y}}

\newcommand{\E}{\mathbb{E}}
\newcommand{\bfX}{\mathbf{X}}
\newcommand{\bfV}{\mathbf{V}}

% END
