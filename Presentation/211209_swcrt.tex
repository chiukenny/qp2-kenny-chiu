\documentclass[9pt]{beamer}

\usepackage{amsmath}
\usepackage{amssymb}
\usepackage{mathtools}
\usepackage{bm}
\usepackage{amsthm}
\usepackage{thmtools}
\usepackage{xspace}
\usepackage{fancyvrb}
\usepackage{graphicx}
\usepackage{booktabs}
\usepackage{caption}
\usepackage[ruled,vlined]{algorithm2e}
%\usepackage[citestyle=apa,bibstyle=authoryear]{biblatex}
\usepackage[numbers]{natbib}

\definecolor{UBCblue}{rgb}{0.04706, 0.13725, 0.26667}
% This file provides examples of some useful macros for typesetting
% dissertations.  None of the macros defined here are necessary beyond
% for the template documentation, so feel free to change, remove, and add
% your own definitions.
%
% We recommend that you define macros to separate the semantics
% of the things you write from how they are presented.  For example,
% you'll see definitions below for a macro \file{}: by using
% \file{} consistently in the text, we can change how filenames
% are typeset simply by changing the definition of \file{} in
% this file.

%% The following is a directive for TeXShop to indicate the main file
%%!TEX root = 210824_entromin.tex

\newcommand{\NA}{\textsc{n/a}}	% for "not applicable"
\newcommand{\eg}{e.g.,\ }	% proper form of examples (\eg a, b, c)
\newcommand{\ie}{i.e.,\ }	% proper form for that is (\ie a, b, c)
\newcommand{\etal}{\emph{et al}}

% Some useful macros for typesetting terms.
\newcommand{\file}[1]{\texttt{#1}}
\newcommand{\class}[1]{\texttt{#1}}
\newcommand{\latexpackage}[1]{\href{http://www.ctan.org/macros/latex/contrib/#1}{\texttt{#1}}}
\newcommand{\latexmiscpackage}[1]{\href{http://www.ctan.org/macros/latex/contrib/misc/#1.sty}{\texttt{#1}}}
\newcommand{\env}[1]{\texttt{#1}}
\newcommand{\BibTeX}{Bib\TeX}

% Define a command \doi{} to typeset a digital object identifier (DOI).
% Note: if the following definition raise an error, then you likely
% have an ancient version of url.sty.  Either find a more recent version
% (3.1 or later work fine) and simply copy it into this directory,  or
% comment out the following two lines and uncomment the third.
\DeclareUrlCommand\DOI{}
\newcommand{\doi}[1]{\href{http://dx.doi.org/#1}{\DOI{doi:#1}}}
%\newcommand{\doi}[1]{\href{http://dx.doi.org/#1}{doi:#1}}

% Useful macro to reference an online document with a hyperlink
% as well with the URL explicitly listed in a footnote
% #1: the URL
% #2: the anchoring text
\newcommand{\webref}[2]{\href{#1}{#2}\footnote{\url{#1}}}

% epigraph is a nice environment for typesetting quotations
\makeatletter
\newenvironment{epigraph}{%
	\begin{flushright}
	\begin{minipage}{\columnwidth-0.75in}
	\begin{flushright}
	\@ifundefined{singlespacing}{}{\singlespacing}%
    }{
	\end{flushright}
	\end{minipage}
	\end{flushright}}
\makeatother

% \FIXME{} is a useful macro for noting things needing to be changed.
% The following definition will also output a warning to the console
\newcommand{\FIXME}[1]{\typeout{**FIXME** #1}\textbf{[FIXME: #1]}}

% Theorems
\declaretheoremstyle[%
  spaceabove=\topsep,%
  spacebelow=\topsep,%
  headfont=\bfseries,%
  headpunct={.},%
  postheadspace=.5em,%
  bodyfont=\itshape,%
  headformat=\NAME~\NUMBER%
]{namelessTheorem}
\declaretheorem[style=namelessTheorem,name=Theorem]{btheorem}
\declaretheorem[style=namelessTheorem,sibling=theorem,name=Proposition]{bproposition}
\declaretheorem[style=namelessTheorem,sibling=theorem,name=Lemma]{blemma}
\declaretheorem[style=namelessTheorem,sibling=theorem,name=Corollary]{bcorollary}

\declaretheoremstyle[%
  spaceabove=\topsep,%
  spacebelow=\topsep,%
  headfont=\bfseries,%
  headpunct={},%
  postheadspace=.5em,%
  bodyfont=\itshape,%
  headformat=\NAME~\NUMBER%
]{namelessNoDotTheorem}
\declaretheorem[style=namelessNoDotTheorem,sibling=theorem,name=Proposition]{bpropositionnd}

% Definitions
\declaretheoremstyle[%
  spaceabove=\topsep,%
  spacebelow=\topsep,%
  headfont=\bfseries,%
  headpunct={.},%
  postheadspace=.5em,%
  headformat=\NAME~\NUMBER,%
  qed=$\lrcorner$
]{namelessDefinition}
\declaretheorem[style=namelessDefinition,name=Definition]{bdefinition}

% Assumptions
%\theoremstyle{definition}
\declaretheoremstyle[%
  spaceabove=\topsep,%
  spacebelow=\topsep,%
  headfont=\bfseries,%
  headpunct={.},%
  postheadspace=.5em
]{assumptionBlock}
\declaretheorem[style=assumptionBlock,name=Assumption]{assumption}
%\newtheorem{assumption}{Assumption}

\newcommand{\botsep}{\vspace{4pt}}


% Math commands
\newcommand{\Ebb}{\mathbb{E}\xspace}
\newcommand{\Nbb}{\mathbb{N}\xspace}
\newcommand{\Pbb}{\mathbb{P}\xspace}
\newcommand{\Rbb}{\mathbb{R}\xspace}

\newcommand{\Bcal}{\mathcal{B}\xspace}
\newcommand{\Ocal}{\mathcal{O}\xspace}
\newcommand{\Pcal}{\mathcal{P}\xspace}
\newcommand{\Qcal}{\mathcal{Q}\xspace}
\newcommand{\Rcal}{\mathcal{R}\xspace}
\newcommand{\Scal}{\mathcal{S}\xspace}
\newcommand{\Ucal}{\mathcal{U}\xspace}
\newcommand{\Vcal}{\mathcal{V}\xspace}
\newcommand{\Xcal}{\mathcal{X}\xspace}

\newcommand{\Ascr}{\mathscr{A}\xspace}
\newcommand{\Bscr}{\mathscr{B}\xspace}
\newcommand{\Dscr}{\mathscr{D}\xspace}
\newcommand{\Fscr}{\mathscr{F}\xspace}
\newcommand{\Lscr}{\mathscr{L}\xspace}
\newcommand{\Mscr}{\mathscr{M}\xspace}
\newcommand{\Oscr}{\mathscr{O}\xspace}
\newcommand{\Vscr}{\mathscr{V}\xspace}
\newcommand{\Xscr}{\mathscr{X}\xspace}
\newcommand{\Yscr}{\mathscr{Y}\xspace}
\newcommand{\Zscr}{\mathscr{Z}\xspace}

\DeclareMathOperator*{\argmax}{arg\,max\;}
\DeclareMathOperator*{\argmin}{arg\,min\;}
\DeclareMathOperator*{\deq}{\;\overset{\text{d}}{=}\;}
\DeclareMathOperator*{\aseq}{\;\overset{\text{a.s.}}{=}\;}

\newcommand{\ballset}{\Bcal}
\newcommand{\orthset}{\Ocal}
\newcommand{\permset}{\Pcal}
\newcommand{\doubstocset}{\Qcal}

\newcommand{\bigO}{\Oscr}

\newcommand{\ones}{\mathbf{1}}
\newcommand{\hatmu}{\widehat{\mu}}
\newcommand{\hatB}{\widehat{B}}
\newcommand{\hatD}{\widehat{D}}
\newcommand{\hatU}{\widehat{U}}
\newcommand{\hatV}{\widehat{V}}
\newcommand{\hatY}{\widehat{Y}}
\newcommand{\hatZ}{\widehat{Z}}
\newcommand{\tildeA}{\widetilde{A}}
\newcommand{\tildeR}{\widetilde{R}}
\newcommand{\tildemu}{\widetilde{\mu}}

\newcommand{\todo}{\textcolor{red}{TODO}}

% Names
\newcommand{\R}{\textsc{R}\xspace}
\newcommand{\EntrTwo}{Entromin$_2$\xspace}

% END

%% defs.tex
% this is where you define custom notation, commands, etc.


%%
% full alphabets of different styles
%%

% bf series
\def\bfA{\mathbf{A}}
\def\bfB{\mathbf{B}}
\def\bfC{\mathbf{C}}
\def\bfD{\mathbf{D}}
\def\bfE{\mathbf{E}}
\def\bfF{\mathbf{F}}
\def\bfG{\mathbf{G}}
\def\bfH{\mathbf{H}}
\def\bfI{\mathbf{I}}
\def\bfJ{\mathbf{J}}
\def\bfK{\mathbf{K}}
\def\bfL{\mathbf{L}}
\def\bfM{\mathbf{M}}
\def\bfN{\mathbf{N}}
\def\bfO{\mathbf{O}}
\def\bfP{\mathbf{P}}
\def\bfQ{\mathbf{Q}}
\def\bfR{\mathbf{R}}
\def\bfS{\mathbf{S}}
\def\bfT{\mathbf{T}}
\def\bfU{\mathbf{U}}
\def\bfV{\mathbf{V}}
\def\bfW{\mathbf{W}}
\def\bfX{\mathbf{X}}
\def\bfY{\mathbf{Y}}
\def\bfZ{\mathbf{Z}}

\def\bfb{\mathbf{b}}
\def\bfe{\mathbf{e}}
\def\bft{\mathbf{t}}
\def\bfy{\mathbf{y}}

\def\bfom{\bm{\omega}}
\def\bfOm{\bm{\Omega}}

% bb series
\def\bbA{\mathbb{A}}
\def\bbB{\mathbb{B}}
\def\bbC{\mathbb{C}}
\def\bbD{\mathbb{D}}
\def\bbE{\mathbb{E}}
\def\bbF{\mathbb{F}}
\def\bbG{\mathbb{G}}
\def\bbH{\mathbb{H}}
\def\bbI{\mathbb{I}}
\def\bbJ{\mathbb{J}}
\def\bbK{\mathbb{K}}
\def\bbL{\mathbb{L}}
\def\bbM{\mathbb{M}}
\def\bbN{\mathbb{N}}
\def\bbO{\mathbb{O}}
\def\bbP{\mathbb{P}}
\def\bbQ{\mathbb{Q}}
\def\bbR{\mathbb{R}}
\def\bbS{\mathbb{S}}
\def\bbT{\mathbb{T}}
\def\bbU{\mathbb{U}}
\def\bbV{\mathbb{V}}
\def\bbW{\mathbb{W}}
\def\bbX{\mathbb{X}}
\def\bbY{\mathbb{Y}}
\def\bbZ{\mathbb{Z}}

% cal series
\def\calA{\mathcal{A}}
\def\calB{\mathcal{B}}
\def\calC{\mathcal{C}}
\def\calD{\mathcal{D}}
\def\calE{\mathcal{E}}
\def\calF{\mathcal{F}}
\def\calG{\mathcal{G}}
\def\calH{\mathcal{H}}
\def\calI{\mathcal{I}}
\def\calJ{\mathcal{J}}
\def\calK{\mathcal{K}}
\def\calL{\mathcal{L}}
\def\calM{\mathcal{M}}
\def\calN{\mathcal{N}}
\def\calO{\mathcal{O}}
\def\calP{\mathcal{P}}
\def\calQ{\mathcal{Q}}
\def\calR{\mathcal{R}}
\def\calS{\mathcal{S}}
\def\calT{\mathcal{T}}
\def\calU{\mathcal{U}}
\def\calV{\mathcal{V}}
\def\calW{\mathcal{W}}
\def\calX{\mathcal{X}}
\def\calY{\mathcal{Y}}
\def\calZ{\mathcal{Z}}

\def\tildeU{\widetilde{U}}
\def\hatb{\widehat{\bfb}}
\def\hatOm{\widehat{\bfOm}}
\def\hatW{\widehat{\bfW}}
\def\hatmu{\widehat{\mu}}


%%%%%%%%%%%%%%%%%%%%%%%%%%%%%%%%%%%%%%%%%%%%%%%%%%%%%%%%%%
% text short-cuts
\def\iid{i.i.d.\ } %i.i.d.
\def\ie{i.e.\ }
\def\eg{e.g.\ }
\def\Polya{P\'{o}lya\ }
%%%%%%%%%%%%%%%%%%%%%%%%%%%%%%%%%%%%%%%%%%%%%%%%%%%%%%%%%%

%%%%%%%%%%%%%%%%%%%%%%%%%%%%%%%%%%%%%%%%%%%%%%%%%%%%%%%%%%
% quasi-universal probabilistic and mathematical notation
% my preferences (modulo publication conventions, and clashes like random vectors):
%   vectors: bold, lowercase
%   matrices: bold, uppercase
%   operators: blackboard (e.g., \mathbb{E}), uppercase
%   sets, spaces: calligraphic, uppercase
%   random variables: normal font, uppercase
%   deterministic quantities: normal font, lowercase
%%%%%%%%%%%%%%%%%%%%%%%%%%%%%%%%%%%%%%%%%%%%%%%%%%%%%%%%%%

% operators
\def\P{\bbP} %fundamental probability
\def\E{\bbE} %expectation
% conditional expectation
\DeclarePairedDelimiterX\bigCond[2]{[}{]}{#1 \;\delimsize\vert\; #2}
\newcommand{\conditional}[3][]{\bbE_{#1}\bigCond*{#2}{#3}}
\def\Law{\mathcal{L}} %law; this is by convention in the literature
\def\indicator{\mathds{1}} % indicator function

% sets and groups
\def\borel{\calB} %Borel sets
\def\sigAlg{\calA} %sigma-algebra
\def\filtration{\calF} %filtration
\def\grp{\calG} %group

% binary relations
\def\condind{{\perp\!\!\!\perp}} %independence/conditional independence
\def\equdist{\stackrel{\text{\rm\tiny d}}{=}} %equal in distribution
\def\equas{\stackrel{\text{\rm\tiny a.s.}}{=}} %euqal amost surely
\def\simiid{\sim_{\mbox{\tiny iid}}} %sampled i.i.d

% common vectors and matrices
\def\onevec{\mathbf{1}}
\def\iden{\mathbf{I}} % identity matrix
\def\supp{\text{\rm supp}}

% misc
% floor and ceiling
\DeclarePairedDelimiter{\ceilpair}{\lceil}{\rceil}
\DeclarePairedDelimiter{\floor}{\lfloor}{\rfloor}
\newcommand{\argdot}{{\,\vcenter{\hbox{\tiny$\bullet$}}\,}} %generic argument dot

\DeclareMathOperator*{\argmax}{arg\,max}
\DeclareMathOperator*{\argmin}{arg\,min}
%%%%%%%%%%%%%%%%%%%%%%%%%%%%%%%%%%%%%%%%%%%%%%%%%%%%%%%%%%

%%%%%%%%%%%%%%%%%%%%%%%%%%%%%%%%%%%%%%%%%%%%%%%%%%%%%%%%%%
%% some distributions
% continuous
\def\UnifDist{\text{\rm Unif}}
\def\BetaDist{\text{\rm Beta}}
\def\ExpDist{\text{\rm Exp}}
\def\GammaDist{\text{\rm Gamma}}
% \def\GenGammaDist{\text{\rm GGa}} %Generalized Gamma

% discrete
\def\BernDist{\text{\rm Bernoulli}}
\def\BinomDist{\text{\rm Binomial}}
\def\PoissonPlus{\text{\rm Poisson}_{+}}
\def\PoissonDist{\text{\rm Poisson}}
\def\NBPlus{\text{\rm NB}_{+}}
\def\NBDist{\text{\rm NB}}
\def\GeomDist{\text{\rm Geom}}
% \def\CRP{\text{\rm CRP}}
% \def\EGP{\text{\rm EGP}}
% \def\MittagLeffler{\text{\rm ML}}
%%%%%%%%%%%%%%%%%%%%%%%%%%%%%%%%%%%%%%%%%%%%%%%%%%%%%%%%%%

%%%%%%%%%%%%%%%%%%%%%%%%%%%%%%%%%%%%%%%%%%%%%%%%%%%%%%%%%%
% Project-specific notation should go here
% (Because it's at the end of the file, it can overwrite anything that came before.)

%e.g.,
\def\Laplacian{\calL}
%\def\P{\calP}

% combinatorial objects
\def\perm{\sigma} %fixed permutation
\def\Perm{\Sigma} %random permutation
\def\part{\pi} %fixed partition
\def\Part{\Pi} %random partition

% Kernels
\def\MMD{\mathrm{MMD}}
\def\hatMMD{\widehat{\MMD}}
\def\dhatMMD{\widehat{\vphantom{\rule{1.5pt}{5.5pt}}\smash{\hatMMD}}}
\def\d{\mathrm{d}}

\def\xo{x^{(1)}}
\def\xt{x^{(2)}}
\def\yo{y^{(1)}}
\def\yt{y^{(2)}}
\def\go{g^{(1)}}
\def\gtw{g^{(2)}}

% Theorems
\newtheorem{theorem}{Theorem}[section]
\newtheorem{proposition}[theorem]{Proposition}
\newtheorem{corollary}[theorem]{Corollary}
\renewcommand{\thetheorem}{\arabic{theorem}}


\newcommand{\mre}{\textsc{MRE}\xspace}
\newcommand{\Var}{\mathrm{Var}}
\newcommand{\bias}{\mathrm{bias}}
\newcommand{\barY}{\bar{Y}}

\newcommand{\T}{\top}

\newcommand\sbullet[1][.5]{\mathbin{\vcenter{\hbox{\scalebox{.75}{$\bullet$}}}}}

%%%%%%%%%%%%%%%%%%%%%%%%%%%%%%%%%%%%%%%%%%%%%%%%%%%%%%%%%%

\newcommand{\todo}{{\color{red}{TODO}}}

\usepackage{hyperref}

% Bibliography
\setlength{\bibsep}{5pt plus 0.3ex}
%\bibliography{../thesis/refs.bib}
%\setbeamertemplate{bibliography item}{}
%\setlength\bibitemsep{1.5em}
%\renewcommand\bibfont{\footnotesize}

% Theme
\usetheme{Madrid}
\useoutertheme{infolines}
\useinnertheme{circles}
\usecolortheme[named=UBCblue]{structure}

\beamertemplatenavigationsymbolsempty

\setbeamerfont{footnote}{size=\tiny}
\newcommand\blfootnote[1]{%
  \begingroup
  \renewcommand\thefootnote{}\footnote{#1}%
  \addtocounter{footnote}{-1}%
  \endgroup
}

\usefonttheme[onlymath]{serif}
\captionsetup{font=scriptsize,labelfont=scriptsize}
%\captionsetup[subfigure]{labelformat=empty}

% Show section slides
\AtBeginSection[]{
  \begin{frame}
  \vfill
  \centering
  \begin{beamercolorbox}[sep=8pt,center,shadow=true,rounded=true]{title}
    \usebeamerfont{title}\insertsectionhead\par%
  \end{beamercolorbox}
  \vfill
  \end{frame}
}


% Tables
\renewcommand{\arraystretch}{1.1}
\setlength\heavyrulewidth{0.1em}


% Algorithms
\makeatletter
% Booktab style
\renewcommand*{\@algocf@pre@ruled}{\hrule height\heavyrulewidth depth0pt \kern\belowrulesep}
\renewcommand*{\algocf@caption@ruled}{\box\algocf@capbox\kern\aboverulesep\hrule height\lightrulewidth\kern\belowrulesep}
\renewcommand*{\@algocf@post@ruled}{\kern\aboverulesep\hrule height\heavyrulewidth\relax}
\makeatother


% Source path
%\newcommand{\src}{../thesis}


\title{Stepped Wedge Cluster Randomized Trials}
\author[Kenny Chiu]{\large Kenny Chiu\\\vspace{0.5em} \footnotesize Supervising Faculty: John Petkau}
\institute[]{\footnotesize The University of British Columbia\\Department of Statistics}
\date[August 24, 2021]{\large STAT 548 Qualifying Paper 2\\\normalsize Oral Presentation\\\vspace{0.5em}\footnotesize December 9, 2021}


\begin{document}


\begin{frame}
\titlepage
\end{frame}



\begin{frame}{Outline}
\tableofcontents
\end{frame}


\section{Introduction}



\begin{frame}{Background}
Paper by \citet{Hussey:2007} can be viewed as entry point to stepped wedge cluster randomized trials (SW-CRT)
\begin{itemize}

\item
Provides an overview of motivation, design and analysis of SW-CRTs

\item
Focuses on technical aspects of practical interest such as power and estimators

\item
Presents ideas in an accessible and succinct format

\end{itemize}
\vspace{2em}

Main limitations and weaknesses from our perspective:
\begin{enumerate}

\item
Limited breadth: discussion is restricted to primarily one SW-CRT setting

\item
Minimal depth: technical details are only briefly explained or omitted entirely

\item
Writing: unclear which aspects are novel; some typos and/or errors

\end{enumerate}
\end{frame}



\begin{frame}{Objective}
Our main goal is to address the limitations of \citet{Hussey:2007}:
\vspace{1.5em}

\begin{enumerate}
\setlength\itemsep{1.5em}

\item
Address missed technical details, explanations and derivations

\item
Clarify their simulation procedure and attempt to replicate their simulation results

\item
Discuss extensions to their basic model for different SW-CRT settings

\end{enumerate}

\end{frame}



\section{Analysis of stepped wedge cluster randomized trials}



\begin{frame}{Assumed SW-CRT setting}
\textit{Washington State Community Expedited Partner Treatment (EPT) Trial}:
\begin{itemize}

\item
Hypothesis: EPT public health programs decrease prevalence of chlamydia and incidence of gonorrhea in young women

\item
Method: Program implemented in 23 local health jurisdictions (LHJ) in 4 waves; primary outcomes were prevalence (incidence) of chlamydia (gonorrhea) in tested women
\end{itemize}
\vspace{2em}

Primary SW-CRT setting based on EPT trial that \citet{Hussey:2007} work under:
\begin{itemize}

\item
SW-CRT with $I=24$ clusters and $T=5$ measured time points

\item
Cross-sectional design with $N=100$ units at each cluster-time

\end{itemize}
\end{frame}



\begin{frame}{Statistical model}
Individual-level model under assumed SW-CRT setting:
\begin{align*}
Y_{ijk} &= \mu + \alpha_i + \beta_j + X_{ij}\theta + e_{ijk} \\
&= \mu_{ij} + e_{ijk}
\end{align*}
\begin{itemize}

\item
$\mu$ is the mean across clusters and time

\item
$\alpha_i\sim N(0,\tau^2)$ is a random effect for cluster $i\in\{1,\ldots,I\}$

\item
$\beta_j$ is a fixed effect for time point $j\in\{1,\ldots,T-1\}$ ($\beta_T=0$ for identifiability)

\item
$X_{ij}$ is a treatment indicator for cluster $i$ at time $j$ (1 denotes intervention)

\item
$\theta$ is the treatment effect of interest

\item
$e_{ijk}\sim N(0,\sigma^2)$ are i.i.d. noise

\end{itemize}
\end{frame}



\begin{frame}{Methods for estimating treatment effect $\theta$ \citep{Hussey:2007}} \label{slide:methods}

\begin{enumerate}
\setlength\itemsep{2em}

\item
Within-cluster estimator
\begin{itemize}
\setlength\itemsep{0.5em}
\item
Consistent if no time effects ($\beta_j=0$ for all $j$); biased otherwise [\hyperlink{apx:withincluster}{A1}]
\end{itemize}

\item
Linear mixed effects model (LMM) via weighted least squares (WLS)
\begin{itemize}
\setlength\itemsep{0.5em}
%\item
%Cluster-level analysis where individual-level data are first aggregated
\item
Useful if $\tau^2$ and $\sigma^2$ known or clusters roughly equal sized; loss of power otherwise due to misspecified weights
\item
More efficient than within-cluster estimator if no time effects; note \citet{Liao:2015} found an error in \citeauthor{Hussey:2007}' relative efficiency [\hyperlink{apx:efficiency}{A2}]
\end{itemize}

\item
Generalized linear mixed effects model (GLMM)
\begin{itemize}
\setlength\itemsep{0.5em}
\item
Weights are appropriately weighted even if variance components unknown
\item
Link function allows choice of how expected response is modeled
\end{itemize}

\item
Generalized estimating equations (GEE)
\begin{itemize}
\setlength\itemsep{0.5em}
\item
Consistent even if correlation structure misspecified as long as mean is correctly specified
\end{itemize}

\end{enumerate}
\end{frame}



\begin{frame}{Power calculation} \label{slide:power}
\citet{Hussey:2007} prescribe using a Wald test to test $H_0:\theta=0$
\begin{itemize}
\item
Power for a test of size $\alpha$ is approximately
\[
\Phi\left(\frac{\theta_a}{\sqrt{\Var(\hat{\theta})}} - Z_{1-\frac{\alpha}{2}}\right)
\]
where $\Phi$ is the cumulative distribution function of a standard normal [\hyperlink{apx:power}{A3}]
\end{itemize}
\vspace{2em}

\citet{Hussey:2007} also show that
\begin{itemize}

\item
power is maximized when each cluster crosses over at its own time point [\hyperlink{apx:timepoints}{A4}]

\item
delays in treatment effect decreases power [\hyperlink{apx:delay}{A5}]

\end{itemize}
\end{frame}



\section{Investigation of simulation study}



\begin{frame}{Procedure}
\end{frame}



\begin{frame}{Results}
\end{frame}



\section{Extensions to basic model}



\section{Conclusion}



\begin{frame}{Multiple factor analysis}
\begin{itemize}
\setlength\itemsep{2em}

\visible<1->{
\item
Factor analysis: estimate latent factors underlying observed data

\item
\textit{Principal Component Analysis}: given data matrix $A\in\Rbb^{n\times d}$, returns scaled \\\textit{loadings} $V\in\Rbb^{d\times d}$ and \textit{principal components} (PCs) $S\in\Rbb^{n\times d}$ s.t.
\[
S=AV
\]
$\Rightarrow$ Represent and estimate factors by the leading $k\leq d$ PCs

\item
PCs are (orthogonal) vectors---difficult to interpret if there are many coefficients
}

\visible<2->{
\item
\citet{Thurstone:1935}: rotate PCs to make coefficients as sparse as possible (\ie $\approx0$)
\iffalse
\vspace{0.75em}
\begin{itemize}
\setlength\itemsep{0.75em}

\item
Data representation invariant to orthogonal rotations $R$: $A=SV^T=(SR)(VR)^T$

\item
Sparsity constraint breaks symmetry in set of equally-good factor representations

\end{itemize}
\fi
}
\end{itemize}
\end{frame}



\begin{frame}[allowframebreaks]{References}
%\printbibliography
\bibliographystyle{apalike}
\bibliography{../report/refs}
\end{frame}



\setbeamertemplate{footline}{}



%\addtocounter{framenumber}{-1}
\begin{frame}{Within-cluster estimator [\hyperlink{slide:methods}{S\ref{slide:methods}}]} \label{apx:withincluster}
\end{frame}



%\addtocounter{framenumber}{-1}
\begin{frame}{Relative efficiency of within-cluster to WLS [\hyperlink{slide:methods}{S\ref{slide:methods}}]} \label{apx:efficiency}
\end{frame}



%\addtocounter{framenumber}{-1}
\begin{frame}{Approximate power [\hyperlink{slide:power}{S\ref{slide:power}}]} \label{apx:power}
\end{frame}



%\addtocounter{framenumber}{-1}
\begin{frame}{Reduced time points [\hyperlink{slide:power}{S\ref{slide:power}}]} \label{apx:timepoints}
\end{frame}



%\addtocounter{framenumber}{-1}
\begin{frame}{Delay in treatment effect [\hyperlink{slide:power}{S\ref{slide:power}}]} \label{apx:delay}
\end{frame}



\end{document}