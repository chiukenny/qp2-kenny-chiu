\documentclass[9pt]{beamer}

\usepackage{amsmath}
\usepackage{amssymb}
\usepackage{mathtools}
\usepackage{bm}
\usepackage{amsthm}
\usepackage{thmtools}
\usepackage{xspace}
\usepackage{fancyvrb}
\usepackage{graphicx}
\usepackage{booktabs}
\usepackage{caption}
\usepackage{multirow}
\usepackage[ruled,vlined]{algorithm2e}
%\usepackage[citestyle=apa,bibstyle=authoryear]{biblatex}
\usepackage[numbers]{natbib}

\definecolor{UBCblue}{rgb}{0.04706, 0.13725, 0.26667}
% This file provides examples of some useful macros for typesetting
% dissertations.  None of the macros defined here are necessary beyond
% for the template documentation, so feel free to change, remove, and add
% your own definitions.
%
% We recommend that you define macros to separate the semantics
% of the things you write from how they are presented.  For example,
% you'll see definitions below for a macro \file{}: by using
% \file{} consistently in the text, we can change how filenames
% are typeset simply by changing the definition of \file{} in
% this file.

%% The following is a directive for TeXShop to indicate the main file
%%!TEX root = 210824_entromin.tex

\newcommand{\NA}{\textsc{n/a}}	% for "not applicable"
\newcommand{\eg}{e.g.,\ }	% proper form of examples (\eg a, b, c)
\newcommand{\ie}{i.e.,\ }	% proper form for that is (\ie a, b, c)
\newcommand{\etal}{\emph{et al}}

% Some useful macros for typesetting terms.
\newcommand{\file}[1]{\texttt{#1}}
\newcommand{\class}[1]{\texttt{#1}}
\newcommand{\latexpackage}[1]{\href{http://www.ctan.org/macros/latex/contrib/#1}{\texttt{#1}}}
\newcommand{\latexmiscpackage}[1]{\href{http://www.ctan.org/macros/latex/contrib/misc/#1.sty}{\texttt{#1}}}
\newcommand{\env}[1]{\texttt{#1}}
\newcommand{\BibTeX}{Bib\TeX}

% Define a command \doi{} to typeset a digital object identifier (DOI).
% Note: if the following definition raise an error, then you likely
% have an ancient version of url.sty.  Either find a more recent version
% (3.1 or later work fine) and simply copy it into this directory,  or
% comment out the following two lines and uncomment the third.
\DeclareUrlCommand\DOI{}
\newcommand{\doi}[1]{\href{http://dx.doi.org/#1}{\DOI{doi:#1}}}
%\newcommand{\doi}[1]{\href{http://dx.doi.org/#1}{doi:#1}}

% Useful macro to reference an online document with a hyperlink
% as well with the URL explicitly listed in a footnote
% #1: the URL
% #2: the anchoring text
\newcommand{\webref}[2]{\href{#1}{#2}\footnote{\url{#1}}}

% epigraph is a nice environment for typesetting quotations
\makeatletter
\newenvironment{epigraph}{%
	\begin{flushright}
	\begin{minipage}{\columnwidth-0.75in}
	\begin{flushright}
	\@ifundefined{singlespacing}{}{\singlespacing}%
    }{
	\end{flushright}
	\end{minipage}
	\end{flushright}}
\makeatother

% \FIXME{} is a useful macro for noting things needing to be changed.
% The following definition will also output a warning to the console
\newcommand{\FIXME}[1]{\typeout{**FIXME** #1}\textbf{[FIXME: #1]}}

% Theorems
\declaretheoremstyle[%
  spaceabove=\topsep,%
  spacebelow=\topsep,%
  headfont=\bfseries,%
  headpunct={.},%
  postheadspace=.5em,%
  bodyfont=\itshape,%
  headformat=\NAME~\NUMBER%
]{namelessTheorem}
\declaretheorem[style=namelessTheorem,name=Theorem]{btheorem}
\declaretheorem[style=namelessTheorem,sibling=theorem,name=Proposition]{bproposition}
\declaretheorem[style=namelessTheorem,sibling=theorem,name=Lemma]{blemma}
\declaretheorem[style=namelessTheorem,sibling=theorem,name=Corollary]{bcorollary}

\declaretheoremstyle[%
  spaceabove=\topsep,%
  spacebelow=\topsep,%
  headfont=\bfseries,%
  headpunct={},%
  postheadspace=.5em,%
  bodyfont=\itshape,%
  headformat=\NAME~\NUMBER%
]{namelessNoDotTheorem}
\declaretheorem[style=namelessNoDotTheorem,sibling=theorem,name=Proposition]{bpropositionnd}

% Definitions
\declaretheoremstyle[%
  spaceabove=\topsep,%
  spacebelow=\topsep,%
  headfont=\bfseries,%
  headpunct={.},%
  postheadspace=.5em,%
  headformat=\NAME~\NUMBER,%
  qed=$\lrcorner$
]{namelessDefinition}
\declaretheorem[style=namelessDefinition,name=Definition]{bdefinition}

% Assumptions
%\theoremstyle{definition}
\declaretheoremstyle[%
  spaceabove=\topsep,%
  spacebelow=\topsep,%
  headfont=\bfseries,%
  headpunct={.},%
  postheadspace=.5em
]{assumptionBlock}
\declaretheorem[style=assumptionBlock,name=Assumption]{assumption}
%\newtheorem{assumption}{Assumption}

\newcommand{\botsep}{\vspace{4pt}}


% Math commands
\newcommand{\Ebb}{\mathbb{E}\xspace}
\newcommand{\Nbb}{\mathbb{N}\xspace}
\newcommand{\Pbb}{\mathbb{P}\xspace}
\newcommand{\Rbb}{\mathbb{R}\xspace}

\newcommand{\Bcal}{\mathcal{B}\xspace}
\newcommand{\Ocal}{\mathcal{O}\xspace}
\newcommand{\Pcal}{\mathcal{P}\xspace}
\newcommand{\Qcal}{\mathcal{Q}\xspace}
\newcommand{\Rcal}{\mathcal{R}\xspace}
\newcommand{\Scal}{\mathcal{S}\xspace}
\newcommand{\Ucal}{\mathcal{U}\xspace}
\newcommand{\Vcal}{\mathcal{V}\xspace}
\newcommand{\Xcal}{\mathcal{X}\xspace}

\newcommand{\Ascr}{\mathscr{A}\xspace}
\newcommand{\Bscr}{\mathscr{B}\xspace}
\newcommand{\Dscr}{\mathscr{D}\xspace}
\newcommand{\Fscr}{\mathscr{F}\xspace}
\newcommand{\Lscr}{\mathscr{L}\xspace}
\newcommand{\Mscr}{\mathscr{M}\xspace}
\newcommand{\Oscr}{\mathscr{O}\xspace}
\newcommand{\Vscr}{\mathscr{V}\xspace}
\newcommand{\Xscr}{\mathscr{X}\xspace}
\newcommand{\Yscr}{\mathscr{Y}\xspace}
\newcommand{\Zscr}{\mathscr{Z}\xspace}

\DeclareMathOperator*{\argmax}{arg\,max\;}
\DeclareMathOperator*{\argmin}{arg\,min\;}
\DeclareMathOperator*{\deq}{\;\overset{\text{d}}{=}\;}
\DeclareMathOperator*{\aseq}{\;\overset{\text{a.s.}}{=}\;}

\newcommand{\ballset}{\Bcal}
\newcommand{\orthset}{\Ocal}
\newcommand{\permset}{\Pcal}
\newcommand{\doubstocset}{\Qcal}

\newcommand{\bigO}{\Oscr}

\newcommand{\ones}{\mathbf{1}}
\newcommand{\hatmu}{\widehat{\mu}}
\newcommand{\hatB}{\widehat{B}}
\newcommand{\hatD}{\widehat{D}}
\newcommand{\hatU}{\widehat{U}}
\newcommand{\hatV}{\widehat{V}}
\newcommand{\hatY}{\widehat{Y}}
\newcommand{\hatZ}{\widehat{Z}}
\newcommand{\tildeA}{\widetilde{A}}
\newcommand{\tildeR}{\widetilde{R}}
\newcommand{\tildemu}{\widetilde{\mu}}

\newcommand{\todo}{\textcolor{red}{TODO}}

% Names
\newcommand{\R}{\textsc{R}\xspace}
\newcommand{\EntrTwo}{Entromin$_2$\xspace}

% END

%% defs.tex
% this is where you define custom notation, commands, etc.


%%
% full alphabets of different styles
%%

% bf series
\def\bfA{\mathbf{A}}
\def\bfB{\mathbf{B}}
\def\bfC{\mathbf{C}}
\def\bfD{\mathbf{D}}
\def\bfE{\mathbf{E}}
\def\bfF{\mathbf{F}}
\def\bfG{\mathbf{G}}
\def\bfH{\mathbf{H}}
\def\bfI{\mathbf{I}}
\def\bfJ{\mathbf{J}}
\def\bfK{\mathbf{K}}
\def\bfL{\mathbf{L}}
\def\bfM{\mathbf{M}}
\def\bfN{\mathbf{N}}
\def\bfO{\mathbf{O}}
\def\bfP{\mathbf{P}}
\def\bfQ{\mathbf{Q}}
\def\bfR{\mathbf{R}}
\def\bfS{\mathbf{S}}
\def\bfT{\mathbf{T}}
\def\bfU{\mathbf{U}}
\def\bfV{\mathbf{V}}
\def\bfW{\mathbf{W}}
\def\bfX{\mathbf{X}}
\def\bfY{\mathbf{Y}}
\def\bfZ{\mathbf{Z}}

\def\bfb{\mathbf{b}}
\def\bfe{\mathbf{e}}
\def\bft{\mathbf{t}}
\def\bfy{\mathbf{y}}

\def\bfom{\bm{\omega}}
\def\bfOm{\bm{\Omega}}

% bb series
\def\bbA{\mathbb{A}}
\def\bbB{\mathbb{B}}
\def\bbC{\mathbb{C}}
\def\bbD{\mathbb{D}}
\def\bbE{\mathbb{E}}
\def\bbF{\mathbb{F}}
\def\bbG{\mathbb{G}}
\def\bbH{\mathbb{H}}
\def\bbI{\mathbb{I}}
\def\bbJ{\mathbb{J}}
\def\bbK{\mathbb{K}}
\def\bbL{\mathbb{L}}
\def\bbM{\mathbb{M}}
\def\bbN{\mathbb{N}}
\def\bbO{\mathbb{O}}
\def\bbP{\mathbb{P}}
\def\bbQ{\mathbb{Q}}
\def\bbR{\mathbb{R}}
\def\bbS{\mathbb{S}}
\def\bbT{\mathbb{T}}
\def\bbU{\mathbb{U}}
\def\bbV{\mathbb{V}}
\def\bbW{\mathbb{W}}
\def\bbX{\mathbb{X}}
\def\bbY{\mathbb{Y}}
\def\bbZ{\mathbb{Z}}

% cal series
\def\calA{\mathcal{A}}
\def\calB{\mathcal{B}}
\def\calC{\mathcal{C}}
\def\calD{\mathcal{D}}
\def\calE{\mathcal{E}}
\def\calF{\mathcal{F}}
\def\calG{\mathcal{G}}
\def\calH{\mathcal{H}}
\def\calI{\mathcal{I}}
\def\calJ{\mathcal{J}}
\def\calK{\mathcal{K}}
\def\calL{\mathcal{L}}
\def\calM{\mathcal{M}}
\def\calN{\mathcal{N}}
\def\calO{\mathcal{O}}
\def\calP{\mathcal{P}}
\def\calQ{\mathcal{Q}}
\def\calR{\mathcal{R}}
\def\calS{\mathcal{S}}
\def\calT{\mathcal{T}}
\def\calU{\mathcal{U}}
\def\calV{\mathcal{V}}
\def\calW{\mathcal{W}}
\def\calX{\mathcal{X}}
\def\calY{\mathcal{Y}}
\def\calZ{\mathcal{Z}}

\def\tildeU{\widetilde{U}}
\def\hatb{\widehat{\bfb}}
\def\hatOm{\widehat{\bfOm}}
\def\hatW{\widehat{\bfW}}
\def\hatmu{\widehat{\mu}}


%%%%%%%%%%%%%%%%%%%%%%%%%%%%%%%%%%%%%%%%%%%%%%%%%%%%%%%%%%
% text short-cuts
\def\iid{i.i.d.\ } %i.i.d.
\def\ie{i.e.\ }
\def\eg{e.g.\ }
\def\Polya{P\'{o}lya\ }
%%%%%%%%%%%%%%%%%%%%%%%%%%%%%%%%%%%%%%%%%%%%%%%%%%%%%%%%%%

%%%%%%%%%%%%%%%%%%%%%%%%%%%%%%%%%%%%%%%%%%%%%%%%%%%%%%%%%%
% quasi-universal probabilistic and mathematical notation
% my preferences (modulo publication conventions, and clashes like random vectors):
%   vectors: bold, lowercase
%   matrices: bold, uppercase
%   operators: blackboard (e.g., \mathbb{E}), uppercase
%   sets, spaces: calligraphic, uppercase
%   random variables: normal font, uppercase
%   deterministic quantities: normal font, lowercase
%%%%%%%%%%%%%%%%%%%%%%%%%%%%%%%%%%%%%%%%%%%%%%%%%%%%%%%%%%

% operators
\def\P{\bbP} %fundamental probability
\def\E{\bbE} %expectation
% conditional expectation
\DeclarePairedDelimiterX\bigCond[2]{[}{]}{#1 \;\delimsize\vert\; #2}
\newcommand{\conditional}[3][]{\bbE_{#1}\bigCond*{#2}{#3}}
\def\Law{\mathcal{L}} %law; this is by convention in the literature
\def\indicator{\mathds{1}} % indicator function

% sets and groups
\def\borel{\calB} %Borel sets
\def\sigAlg{\calA} %sigma-algebra
\def\filtration{\calF} %filtration
\def\grp{\calG} %group

% binary relations
\def\condind{{\perp\!\!\!\perp}} %independence/conditional independence
\def\equdist{\stackrel{\text{\rm\tiny d}}{=}} %equal in distribution
\def\equas{\stackrel{\text{\rm\tiny a.s.}}{=}} %euqal amost surely
\def\simiid{\sim_{\mbox{\tiny iid}}} %sampled i.i.d

% common vectors and matrices
\def\onevec{\mathbf{1}}
\def\iden{\mathbf{I}} % identity matrix
\def\supp{\text{\rm supp}}

% misc
% floor and ceiling
\DeclarePairedDelimiter{\ceilpair}{\lceil}{\rceil}
\DeclarePairedDelimiter{\floor}{\lfloor}{\rfloor}
\newcommand{\argdot}{{\,\vcenter{\hbox{\tiny$\bullet$}}\,}} %generic argument dot

\DeclareMathOperator*{\argmax}{arg\,max}
\DeclareMathOperator*{\argmin}{arg\,min}
%%%%%%%%%%%%%%%%%%%%%%%%%%%%%%%%%%%%%%%%%%%%%%%%%%%%%%%%%%

%%%%%%%%%%%%%%%%%%%%%%%%%%%%%%%%%%%%%%%%%%%%%%%%%%%%%%%%%%
%% some distributions
% continuous
\def\UnifDist{\text{\rm Unif}}
\def\BetaDist{\text{\rm Beta}}
\def\ExpDist{\text{\rm Exp}}
\def\GammaDist{\text{\rm Gamma}}
% \def\GenGammaDist{\text{\rm GGa}} %Generalized Gamma

% discrete
\def\BernDist{\text{\rm Bernoulli}}
\def\BinomDist{\text{\rm Binomial}}
\def\PoissonPlus{\text{\rm Poisson}_{+}}
\def\PoissonDist{\text{\rm Poisson}}
\def\NBPlus{\text{\rm NB}_{+}}
\def\NBDist{\text{\rm NB}}
\def\GeomDist{\text{\rm Geom}}
% \def\CRP{\text{\rm CRP}}
% \def\EGP{\text{\rm EGP}}
% \def\MittagLeffler{\text{\rm ML}}
%%%%%%%%%%%%%%%%%%%%%%%%%%%%%%%%%%%%%%%%%%%%%%%%%%%%%%%%%%

%%%%%%%%%%%%%%%%%%%%%%%%%%%%%%%%%%%%%%%%%%%%%%%%%%%%%%%%%%
% Project-specific notation should go here
% (Because it's at the end of the file, it can overwrite anything that came before.)

%e.g.,
\def\Laplacian{\calL}
%\def\P{\calP}

% combinatorial objects
\def\perm{\sigma} %fixed permutation
\def\Perm{\Sigma} %random permutation
\def\part{\pi} %fixed partition
\def\Part{\Pi} %random partition

% Kernels
\def\MMD{\mathrm{MMD}}
\def\hatMMD{\widehat{\MMD}}
\def\dhatMMD{\widehat{\vphantom{\rule{1.5pt}{5.5pt}}\smash{\hatMMD}}}
\def\d{\mathrm{d}}

\def\xo{x^{(1)}}
\def\xt{x^{(2)}}
\def\yo{y^{(1)}}
\def\yt{y^{(2)}}
\def\go{g^{(1)}}
\def\gtw{g^{(2)}}

% Theorems
\newtheorem{theorem}{Theorem}[section]
\newtheorem{proposition}[theorem]{Proposition}
\newtheorem{corollary}[theorem]{Corollary}
\renewcommand{\thetheorem}{\arabic{theorem}}


\newcommand{\mre}{\textsc{MRE}\xspace}
\newcommand{\Var}{\mathrm{Var}}
\newcommand{\bias}{\mathrm{bias}}
\newcommand{\barY}{\bar{Y}}

\newcommand{\T}{\top}

\newcommand\sbullet[1][.5]{\mathbin{\vcenter{\hbox{\scalebox{.75}{$\bullet$}}}}}

%%%%%%%%%%%%%%%%%%%%%%%%%%%%%%%%%%%%%%%%%%%%%%%%%%%%%%%%%%

\newcommand{\todo}{{\color{red}{TODO}}}

\usepackage{hyperref}

% Bibliography
\setlength{\bibsep}{5pt plus 0.3ex}
%\bibliography{../thesis/refs.bib}
%\setbeamertemplate{bibliography item}{}
%\setlength\bibitemsep{1.5em}
%\renewcommand\bibfont{\footnotesize}

% Theme
\usetheme{Madrid}
\useoutertheme{infolines}
\useinnertheme{circles}
\usecolortheme[named=UBCblue]{structure}

\beamertemplatenavigationsymbolsempty

\setbeamerfont{footnote}{size=\tiny}
\newcommand\blfootnote[1]{%
  \begingroup
  \renewcommand\thefootnote{}\footnote{#1}%
  \addtocounter{footnote}{-1}%
  \endgroup
}

\usefonttheme[onlymath]{serif}
\captionsetup{font=scriptsize,labelfont=scriptsize}
%\captionsetup[subfigure]{labelformat=empty}

% Show section slides
\AtBeginSection[]{
  \begin{frame}
  \vfill
  \centering
  \begin{beamercolorbox}[sep=8pt,center,shadow=true,rounded=true]{title}
    \usebeamerfont{title}\insertsectionhead\par%
  \end{beamercolorbox}
  \vfill
  \end{frame}
}


% Tables
\renewcommand{\arraystretch}{1.1}
\setlength\heavyrulewidth{0.1em}


% Algorithms
\makeatletter
% Booktab style
\renewcommand*{\@algocf@pre@ruled}{\hrule height\heavyrulewidth depth0pt \kern\belowrulesep}
\renewcommand*{\algocf@caption@ruled}{\box\algocf@capbox\kern\aboverulesep\hrule height\lightrulewidth\kern\belowrulesep}
\renewcommand*{\@algocf@post@ruled}{\kern\aboverulesep\hrule height\heavyrulewidth\relax}
\makeatother


% Source path
%\newcommand{\src}{../thesis}


\title{Stepped Wedge Cluster Randomized Trials}
\author[Kenny Chiu]{\large Kenny Chiu\\\vspace{0.5em} \footnotesize QP Supervisor: John Petkau}
\institute[]{\footnotesize The University of British Columbia\\Department of Statistics}
\date[August 24, 2021]{\large STAT 548 Qualifying Paper 2\\\normalsize Oral Presentation\\\vspace{0.5em}\footnotesize December 9, 2021}


\begin{document}


\begin{frame}
\titlepage
\end{frame}



\begin{frame}{Outline}
\tableofcontents
\end{frame}


\section{Introduction}



\begin{frame}{Background}
Paper by \citet{Hussey:2007} can be viewed as entry point to stepped wedge cluster randomized trials (SW-CRT)
\begin{itemize}

\item
Provides an overview of motivation, design and analysis of SW-CRTs

\item
Focuses on technical aspects of practical interest such as power and estimators

\item
Presents ideas in an accessible and succinct format

\end{itemize}
\vspace{2em}

Main limitations and weaknesses from our perspective:
\begin{enumerate}

\item
Limited breadth: discussion is restricted to primarily one SW-CRT setting

\item
Minimal depth: technical details are only briefly explained or omitted entirely

\item
Writing: unclear which aspects are novel; some typos and/or errors

\end{enumerate}
\end{frame}



\begin{frame}{Objective}
Our main goal is to address the limitations of \citet{Hussey:2007}:
\vspace{1.5em}

\begin{enumerate}
\setlength\itemsep{1.5em}

\item
Address missed technical details, explanations and derivations

\item
Clarify their simulation procedure and attempt to replicate their simulation results

\item
Discuss extensions to their basic model for different SW-CRT settings

\end{enumerate}

\end{frame}



\section{Analysis of stepped wedge cluster randomized trials}



\begin{frame}{Assumed SW-CRT setting}
\textit{Washington State Community Expedited Partner Treatment (EPT) Trial}:
\begin{itemize}

\item
Hypothesis: EPT public health programs decrease prevalence of chlamydia and incidence of gonorrhea in young women

\item
Method: program implemented in 23 local health jurisdictions in 4 waves

\item
Primary outcomes: prevalence (incidence) of chlamydia (gonorrhea) in tested women
\end{itemize}
\vspace{2em}

Primary SW-CRT setting based on EPT trial that \citet{Hussey:2007} work under:
\begin{itemize}

\item
SW-CRT with $I=24$ clusters and $T=5$ measured time points

\item
Cross-sectional design with $N=100$ units at each cluster-time

\end{itemize}
\end{frame}



\begin{frame}{Statistical model}
Individual-level model under assumed SW-CRT setting:
\begin{align*}
Y_{ijk} &= \mu + \alpha_i + \beta_j + X_{ij}\theta + e_{ijk} \\
&= \mu_{ij} + e_{ijk}
\end{align*}
\begin{itemize}

\item
$\mu$ is the mean across clusters and time

\item
$\alpha_i\sim N(0,\tau^2)$ is a random effect for cluster $i\in\{1,\ldots,I\}$

\item
$\beta_j$ is a fixed effect for time point $j\in\{1,\ldots,T-1\}$ ($\beta_T=0$ for identifiability)

\item
$X_{ij}$ is a treatment indicator for cluster $i$ at time $j$ (1 denotes intervention)

\item
$\theta$ is the treatment effect of interest

\item
$e_{ijk}\sim N(0,\sigma^2)$ are i.i.d. noise

\end{itemize}
\end{frame}



\begin{frame}{Methods for estimating treatment effect $\theta$ \citep{Hussey:2007}} \label{slide:methods}

\begin{enumerate}
\setlength\itemsep{2em}

\item
Within-cluster estimator
\begin{itemize}
\setlength\itemsep{0.5em}
\item
Consistent if no time effects ($\beta_j=0$ for all $j$); biased otherwise [\hyperlink{apx:withincluster}{\color{red}A1}]
\end{itemize}

\item
Linear mixed effects model (LMM) via weighted least squares (WLS)
\begin{itemize}
\setlength\itemsep{0.5em}
%\item
%Cluster-level analysis where individual-level data are first aggregated
\item
More efficient than within-cluster estimator if no time effects; note \citet{Liao:2015} found an error in \citeauthor{Hussey:2007}' relative efficiency [\hyperlink{apx:efficiency}{\color{red}A2}]
\item
Useful if $\tau^2$ and $\sigma^2$ known or clusters roughly equal sized; loss of power otherwise due to misspecified weights
\end{itemize}

\item
Generalized linear mixed effects model (GLMM)
\begin{itemize}
\setlength\itemsep{0.5em}
\item
Weights are appropriately weighted even if variance components unknown
\item
Link function allows choice of how expected response is modeled
\end{itemize}

\item
Generalized estimating equations (GEE)
\begin{itemize}
\setlength\itemsep{0.5em}
\item
Consistent even if correlation structure misspecified as long as mean is correctly specified
\end{itemize}

\end{enumerate}
\end{frame}



\begin{frame}{Power calculation} \label{slide:power}
\citet{Hussey:2007} prescribe using a Wald test to test $H_0:\theta=0$
\begin{itemize}
\item
Power for two-sided test of size $\alpha$ is approximately
\[
\Phi\left(\frac{\theta_a}{\sqrt{\Var(\hat{\theta})}} - Z_{1-\frac{\alpha}{2}}\right)
\]
where $\Phi$ is the CDF of a standard normal [\hyperlink{apx:power}{\color{red}A3}]
\end{itemize}
\vspace{2em}

\citet{Hussey:2007} also show that
\begin{enumerate}

\item
power is maximized when each cluster crosses over at its own time point [\hyperlink{apx:timepoints}{\color{red}A4}]

\item
delays in treatment effect decreases power [\hyperlink{apx:delay}{\color{red}A5}]

\end{enumerate}
\end{frame}



\section{Investigation of simulation study}



\begin{frame}{Study purpose}
\begin{itemize}
\setlength\itemsep{2em}

\item
\citet{Hussey:2007} conduct a simulation study to compare powers for testing the treatment effect in LMM, GLMM and GEE

\item
Their procedure is unclear based on their description

\item
We aim to clarify details of their procedure by attempting to replicate their results

\end{itemize}
\vspace{2em}

High-level procedure:
\begin{enumerate}
\item
Select configuration (treatment effect $\theta$, equal/unequal cluster sizes)
\item
Simulate 1000 datasets under selected configuration
\item
Estimate power over 1000 models for each of LMM, GLMM and GEE
\end{enumerate}
\end{frame}



\begin{frame}{Data simulation procedure}
Across all configurations, data simulated based on EPT trial
\begin{itemize}

\item
$I=24$, $T=5$, $\mu=0.05$, $\tau^2=0.000225$

\item
Risk ratio (RR) chosen for study determines $\theta=\mu(\text{RR}-1)$

\item
At any time point, total number of individuals across clusters $= 2400$

\end{itemize}
\vspace{2em}

For a single configuration, in each of 1000 simulations:
\begin{enumerate}

\item
Sample cluster effects $\alpha_i\sim N(0,\tau^2)$

\item
Shuffle cluster crossover times $t_1,\ldots,t_I$

\item
Determine cluster sizes
\begin{itemize}
\item
Equal size case: $N_i=100$ for all $i$
\item
Unequal size case: two-step procedure where
\begin{align*}
p &\sim \text{Dirichlet}(1,\ldots,1) \\
\{N_i\}_{i=1}^I &\sim \text{Multinomial}(99I=2376,p) + \begin{bmatrix}1,\ldots,1\end{bmatrix}^T
\end{align*}
\end{itemize}

\item
Sample $N_i$ individuals for cluster $i$ and time $j$ according to $\text{Bernoulli}(p_{ij})$ where
\[
p_{ij} = \max(0,\mu+\alpha_i+\bm{1}(j\geq t_i)\theta)
\]

\end{enumerate}
\end{frame}



\begin{frame}{Model fitting procedure}
Compared models (default function arguments used unless otherwise specified):
\vspace{1em}
\begin{enumerate}
\setlength\itemsep{1em}

\item
LMM (via lme() from nlme):
\[
\mathbb{E}[Y_{ij}|\alpha_i,\beta_j] = \mu+\alpha_i+\beta_j+X_{ij}\theta
\]

\item
GLMM (via glmmPQL() from MASS):
\[
\mathbb{E}[Y_{ijk}|\alpha_i,\beta_j] = \mu+\alpha_i+\beta_j+X_{ij}\theta
\]

\item
GEE (via gee() from gee) with exchangeable correlation structure:
\[
\mathbb{E}[Y_{ijk}|\beta_j] = \mu+\beta_j+X_{ij}\theta
\]

\end{enumerate}
\vspace{2em}

Unclear if \citeauthor{Hussey:2007} use identity or default logit link function for GLMM/GEE
\end{frame}



\begin{frame}{Power calculation procedure}
To estimate power:
\begin{enumerate}

\item
In each simulation, calculate Wald test statistic
\[
W=\frac{\hat{\theta}}{\sqrt{\widehat{\Var}(\hat{\theta})}}
\]
and reject if $|W| > Z_{0.975}$ (critical value of standard normal)

\item
Estimate power = number of rejections / number of non-failing simulations

\end{enumerate}
\vspace{2em}

\citet{Hussey:2007} use two variance estimates:
\begin{enumerate}

\item
``Standard variance'': we interpret as standard error given in function output

\item
Jackknife estimate: we use
\[
\widehat{\Var}(\hat{\theta}) = \frac{1}{M^2}\sum_{i=1}^IN_i^2(\hat{\theta}_i-\hat{\theta}_{\text{JK}})^2
\begin{cases}
\hat{\theta}_i = \frac{M\hat{\theta}(\mathbf{y}) - (M-N_i)\hat{\theta}(\mathbf{y}_{-i})}{N_i} & \text{cluster pseudo-value} \\
\hat{\theta}_{\text{JK}} = \frac{1}{M}\sum_{i=1}^I N_i\hat{\theta}_i & \text{JK estimate of $\theta$}
\end{cases}
\]
where $M=\sum_{i=1}^I N_i$ and $\hat{\theta}(\bullet)$ is estimate based on data $\bullet$
\end{enumerate}
\end{frame}



\begin{frame}{Estimated powers using standard variance}
Original results from \citet{Hussey:2007}:
\begin{table}[h]
\small
\centering
\begin{tabular}{@{}lclllclll@{}}
\toprule
&& \multicolumn{3}{l}{Same cluster sizes} && \multicolumn{3}{l}{Different cluster sizes} \\
\cmidrule{3-5}\cmidrule{7-9}
RR & & LMM & GEE & GLMM && LMM & GEE & GLMM \\
\midrule
1.0 &&  0.056 & 0.084 & 0.076 && 0.048 & 0.095 & 0.069 \\
0.7 && 0.697 & 0.719 & {\color{blue}0.716} && 0.307 & 0.703 & {\color{blue}0.697} \\
0.6 && 0.907 & 0.907 & 0.917 && 0.487 & 0.879 & 0.906 \\
0.5 && 0.988 & 0.990 & 0.992 && {\color{purple}0.625} & 0.982 & 0.986 \\
\bottomrule
\end{tabular}
\end{table}
Our results:
\begin{table}[h]
\small
\centering
\begin{tabular}{@{}llllcll|lllcll@{}}
\toprule
 & \multicolumn{6}{l|}{Same cluster sizes} & \multicolumn{6}{l}{Different cluster sizes} \\
\cmidrule{2-7}\cmidrule{8-13}
& LMM & \multicolumn{2}{l}{GEE} && \multicolumn{2}{l|}{GLMM} & LMM & \multicolumn{2}{l}{GEE\footnote{Estimated over 100 simulations}} && \multicolumn{2}{l}{GLMM} \\
\cmidrule{3-4}\cmidrule{6-7}\cmidrule{9-10}\cmidrule{12-13}
 RR & & id & logit && id & logit & & id & logit && id & logit \\
\midrule
1.0 &  0.050 & 0.089 & 0.081 && 0.066 & 0.052 & 0.062 & 0.11 & 0.10 && 0.058 & 0.053 \\
0.7 & 0.700 & 0.736 & 0.723 && {\color{blue}0.805} & {\color{blue}0.711} & 0.345 & 0.70 & 0.68 && {\color{blue}0.779} & {\color{blue}0.688} \\
0.6 & 0.920 & 0.928 & 0.920 && 0.963 & 0.929 & 0.536 & 0.95 & 0.93 && 0.951 & 0.913 \\
0.5 & 0.981 & 0.983 & 0.982 && 0.994 & 0.985 & {\color{purple}0.719}\footnote{Estimated over 998 non-failing simulations} & 0.99 & 0.97 && 0.997 & 0.985 \\
\bottomrule
\end{tabular}
\end{table}
\end{frame}



\begin{frame}{Estimated powers using jackknife estimate}
Original results from \citet{Hussey:2007}:
\begin{table}[h]
\small
\centering
\begin{tabular}{@{}lclllclll@{}}
\toprule
&& \multicolumn{3}{l}{Same cluster sizes} && \multicolumn{3}{l}{Different cluster sizes} \\
\cmidrule{3-5}\cmidrule{7-9}
RR & & LMM & {\color{lightgray}GEE} & GLMM && LMM & {\color{lightgray}GEE} & GLMM \\
\midrule
1.0 &&  0.057 & {\color{lightgray}0.052} & 0.053 && 0.038 & {\color{lightgray}0.053} & 0.049 \\
0.7 && 0.658 & {\color{lightgray}0.644} & 0.580 && 0.307 & {\color{lightgray}0.577} & 0.559 \\
0.6 && 0.884 & {\color{lightgray}0.866} & 0.820 && 0.503 & {\color{lightgray}0.807} & 0.805 \\
0.5 && 0.984 & {\color{lightgray}0.981} & 0.948 && 0.653 & {\color{lightgray}0.946} & 0.942 \\
\bottomrule
\end{tabular}
\end{table}
Our results (estimated over 100 simulations):
\begin{table}[h]
\small
\centering
\begin{tabular}{@{}lclllclll@{}}
\toprule
&& \multicolumn{3}{l}{Same cluster sizes} && \multicolumn{3}{l}{Different cluster sizes} \\
\cmidrule{3-5}\cmidrule{7-9}
\multirow{2}{*}{\parbox{0.75cm}{Risk ratio}} && LMM & \multicolumn{2}{l}{GLMM} && LMM & \multicolumn{2}{l}{GLMM} \\
\cmidrule{4-5}\cmidrule{8-9}
&& & id & logit && & id & logit \\
\midrule
1.0 && 0.06 & 0.09 & 0.07 && 0.02 & 0.08 & 0.07 \\
0.7 && 0.69 & 0.70 & 0.70 && 0.28\footnote{\label{fn1}Estimated over 99 non-failing simulations} & 0.73 & 0.69 \\
0.6 && 0.90 & 0.95 & 0.91 && 0.61 & 0.93 & 0.89$^{\text{3}}$ \\
0.5 && 1.00 & 1.00 & 1.00 && 0.66 & 0.99$^{\text{3}}$ & 0.93$^{\text{3}}$ \\
\bottomrule
\end{tabular}
\end{table}
\end{frame}



\section{Extensions to basic model}



\begin{frame}{Model extensions I}
1. \textbf{Unequal cluster sizes}
\begin{itemize}

\item
May not be possible to maintain equal number of units in each cluster across time
\begin{itemize}
\item
2017 study: almost half of published trials involved unequal cluster sizes \citep{Kristunas:2017}
\end{itemize}

\item
Generally does not require different model but affects cluster-level variances

\item
Similar to \citet{Hussey:2007}, recent studies \citep{Martin:2019,Ouyang:2020,Kasza:2021,Tian:2021} involving different contexts found results that suggest unequal sizes lead to decreases in power when size is not accounted for

\end{itemize}
\vspace{2em}

2. \textbf{Delayed treatment effect}
\begin{itemize}

\item
Treatment effect may not fully realize over one time period (e.g., \citep{Hughes:2015}, \citep{Agius:2020})

\item
One approach to account for delays is to allow treatment indicator $X_{ij}\in[0,1]$ and take $\theta$ as full treatment effect

\item
Delayed treatment effect decreases power \citep{Hussey:2007}

\item
Modeling delays may be avoided by extending time periods to allow treatment effect to fully realize

\item
Common consideration in SW-CRT literature but not much recent work

\end{itemize}
\end{frame}



\begin{frame}{Model extensions II}
3. \textbf{Non-normal outcomes}
\begin{itemize}

\item
Model with normally-distributed responses may not be reasonable (e.g., binary in EPT trial \citep{Hussey:2007}, 10-point Likert scale in DECIDE-LVAD trial \citep{Allen:2018})

\item
No standard approach for analyzing non-normal outcomes

\item
Power calculation formulas have been proposed for binary and discrete outcomes \citep{Wang:2021,Xia:2021}

\item
One gap in SW-CRT literature appears to be the study of non-normal and non-discrete outcomes (e.g., time)

\end{itemize}
\vspace{2em}

4. \textbf{Cohort designs}
\begin{itemize}

\item
Same participants may be tracked over multiple time periods (open cohort) or throughout trial (closed cohort) (e.g., INSTTEPP trial \citep{Nease:2018})

\item
Account for repeated measurements by adding individual-level random effects $\omega_{ik}$:
\[
Y_{ijk} = \mu + \alpha_i + \beta_j + \omega_{ik} + X_{ij}\theta + e_{ijk}
\]

\item
Recent studies \citep{Hooper:2019,Kasza:2020,Li:2020} examined various cohort designs with differing objectives

\end{itemize}
\end{frame}



\begin{frame}{Model extensions III}
5. \textbf{Hierarchical designs}
\begin{itemize}

\item
There may be multiple levels of clustering (e.g., CHANGE trial \citep{Lescure:2021})

\item
Account for correlation at different levels by adding random effects accordingly, e.g., cluster-time random effects $\omega_{ij}$:
\[
Y_{ijk} = \mu + \alpha_i + \beta_j + \omega_{ij} + X_{ij}\theta + e_{ijk}
\]

\item
General effect of multi-level clustering is the inflation of cluster mean variances at each level \citep{Teerenstra:2019}

\end{itemize}
\vspace{2em}

6. \textbf{Bayesian approaches}
\begin{itemize}

\item
May be desirable to incorporate prior knowledge into model by placing prior distributions on fixed effects and hyperparameters

\item
Bayesian SW-CRT models can be fit using Gibbs sampling \citep{Cunanan:2016}

\item
Recent work found that informative priors reduce calculated sample sizes while bias stays relatively small even if mean is moderately misspecified \citep{Zhan:2021}

\end{itemize}
\end{frame}



\section{Conclusion}



\begin{frame}{Summary}
\begin{itemize}
\setlength\itemsep{2em}

\item
Paper by \citet{Hussey:2007} is accessible but limited in breadth and depth

\item
We provided technical details and derivations where absent in the original work

\item
We obtained similar findings in our simulation study but were unable to replicate exact results

\item
We discussed extensions to the basic model that were not discussed in the original work

\end{itemize}
\end{frame}



\setbeamertemplate{footline}{}



\addtocounter{framenumber}{-3}
\begin{frame}[allowframebreaks]{References}
\footnotesize
%\printbibliography
\bibliographystyle{apalike}
\bibliography{../report/refs}
\end{frame}



\appendix

\addtocounter{framenumber}{-1}
\section{Appendix}



\addtocounter{framenumber}{-1}
\begin{frame}{Within-cluster estimator bias [\hyperlink{slide:methods}{S\ref{slide:methods}}]} \label{apx:withincluster}
Within-cluster estimator \citep{Hussey:2007}:
\[
\tilde{\theta} = \frac{1}{I}\sum_{i=1}^I\left(\frac{\sum_{j=t_i+1}^T\bar{Y}_{ij.}}{T-t_i} - \frac{\sum_{j=1}^{t_i}\bar{Y}_{ij.}}{t_i}\right)
\]
If there are non-trivial time effects, bias is
\begin{align*}
\bias(\tilde{\theta},\theta) &= \E[\tilde{\theta}] - \theta \\
&= \frac{1}{I}\sum_{i=1}^I\left(\frac{\sum_{j=t_i+1}^T(\mu+\alpha_i+\beta_j+\theta)}{T-t_i} - \frac{\sum_{j=1}^{t_i}(\mu+\alpha_i+\beta_j)}{t_i}\right) - \theta \\
&= \frac{1}{I}\sum_{i=1}^I\left(\frac{\sum_{j=1}^T\beta_jX_{ij}}{T-t_i} - \frac{\sum_{j=1}^T\beta_j(1-X_{ij})}{t_i}\right) \\
&= \frac{1}{I}\sum_{j=1}^T\beta_j\sum_{i=1}^I\frac{t_iX_{ij} - (T-t_i)(1-X_{ij})}{t_i(T-t_i)} \\
&= \sum_{j=1}^T\beta_j\sum_{i=1}^I\frac{t_i - T(1-X_{ij})}{It_i(T-t_i)}
\end{align*}
\end{frame}



\addtocounter{framenumber}{-1}
\begin{frame}{Relative efficiency of within-cluster to WLS [\hyperlink{slide:methods}{S\ref{slide:methods}}]} \label{apx:efficiency}
Assume no time effects and let
\begin{align*}
U &= \sum_{i=1}^I\sum_{j=1}^TX_{ij} & V &= \sum_{i=1}^I\left(\sum_{j=1}^TX_{ij}\right)^2
\end{align*}
\vspace{0.5em}

\citet{Hussey:2007} efficiency of WLS estimator $\hat{\theta}$ to within-cluster estimator $\tilde{\theta}$:
\[
\mathrm{efficiency}(\hat{\theta},\tilde{\theta}) = \frac{\Var(\tilde{\theta})}{\Var(\hat{\theta})} = \frac{\sum_{i=1}^I\left(\frac{1}{t_i}+\frac{1}{T-t_i}\right)\left((ITU-U^2)\frac{\sigma^2}{N}+IT(TU-V)\tau^2\right)}{I^3\left(\frac{\sigma^2}{N}+T\tau^2\right)}
\]
\vspace{0.5em}

\citet{Liao:2015} (correct) efficiency:
\[
\mathrm{efficiency}(\hat{\theta},\tilde{\theta}) = \frac{\sum_{i=1}^I\left(\frac{1}{t_i}+\frac{1}{T-t_i}\right)\left((ITU-U^2)\frac{\sigma^2}{N}+IT(TU-V)\tau^2\right)}{I^3T\left(\frac{\sigma^2}{N}+T\tau^2\right)}
\]
\vspace{0.5em}

Efficiency can be shown to be greater or equal to 1
\end{frame}



\addtocounter{framenumber}{-1}
\begin{frame}{Outline of derivation of WLS estimator variance}
Variance of WLS estimator for $\theta$:
\[
\Var(\hat{\theta}) = \frac{IT\frac{\sigma^2}{N}\left(\frac{\sigma^2}{N}+T\tau^2\right)}{(ITU-U^2)\frac{\sigma^2}{N}+IT(UT-V)\tau^2}
\]
Define $IT\times2$ design matrix $\bfX$ and $IT\times IT$ block diagonal $\bfV$ with blocks $\bfV_i$:
\begin{align*}
\bfX &=
\begin{bmatrix}
1 & X_{11} \\
\vdots & \vdots \\
1 & X_{IT}
\end{bmatrix} &
\bfV_i &=
\begin{bmatrix}
\tau^2 + \frac{\sigma^2}{N} & \tau^2 & \ldots & \tau^2 \\
\tau^2 & \ddots & & \vdots \\
\vdots & & \ddots & \tau^2 \\
\tau^2 & \ldots & \tau^2 & \tau^2 + \frac{\sigma^2}{N}
\end{bmatrix}
\end{align*}
Derivation steps:
\begin{enumerate}
\item
Derive $\bfV_i^{-1}$ using Sherman-Morrison formula
\item
Derive $\bfX^T\bfV^{-1}\bfX$ directly from matrix products
\item
Derive $\Var(\hat{\theta}) = \begin{bmatrix}0&1\end{bmatrix}(\bfX^T\bfV^{-1}\bfX)^{-1}\begin{bmatrix}0&1\end{bmatrix}^T$ by inverse of $2\times2$ matrix
\end{enumerate}
\end{frame}



\addtocounter{framenumber}{-1}
\begin{frame}{Approximate power [\hyperlink{slide:power}{S\ref{slide:power}}]} \label{apx:power}
A Wald test uses the test statistic
\[
Z = \frac{\hat{\theta}}{\sqrt{\Var(\hat{\theta})}}
\]
which under the null $H_0:\theta=0$ is standard normally-distributed
\vspace{2em}

Under a simple alternative $H_a:\theta=\theta_a$, the power of a two-tailed test of size $\alpha$ is
\begin{align*}
&\Pbb\left(\left.Z<-Z_{1-\frac{\alpha}{2}}\right|H_a\right) + \Pbb\left(\left.Z>Z_{1-\frac{\alpha}{2}}\right|H_a\right) \\
&= \Phi\left(-\frac{\theta_a}{\sqrt{\Var(\hat{\theta})}}-Z_{1-\frac{\alpha}{2}}\right) + \Phi\left(\frac{\theta_a}{\sqrt{\Var(\hat{\theta})}} - Z_{1-\frac{\alpha}{2}}\right) \\
&\approx \begin{cases}
\Phi\left(\frac{\theta_a}{\sqrt{\Var(\hat{\theta})}} - Z_{1-\frac{\alpha}{2}}\right) & \text{if } \theta_a \gg 0 \\
\Phi\left(-\frac{\theta_a}{\sqrt{\Var(\hat{\theta})}}-Z_{1-\frac{\alpha}{2}}\right) & \text{if } \theta_a \ll 0
\end{cases}
\end{align*}
\end{frame}



\addtocounter{framenumber}{-1}
\begin{frame}{Reduced time points on power [\hyperlink{slide:power}{S\ref{slide:power}}]} \label{apx:timepoints}
By [\hyperlink{apx:power}{A4}], larger $\Var(\hat{\theta})\Rightarrow$ lower power
\vspace{2em}

Variance of within-cluster estimator:
\[
\Var(\tilde{\theta}) = \frac{\sigma^2}{NI^2}\sum_{i=1}^I\left(\frac{1}{T-t_i} + \frac{1}{t_i}\right)
\]
If all but one pair of clusters (WLOG cluster $j$ and $I$) have a unique crossover time,
\begin{align*}
\sum_{i=1}^{I-1}\left(\frac{1}{T-t_i-1} + \frac{1}{t_i}\right) + \frac{1}{T-t_j-1} + \frac{1}{t_j} &= \sum_{i=1}^{I-1}\left(\frac{1}{I-i} + \frac{1}{i}\right) + \frac{1}{I-j} + \frac{1}{j} \\
&> \sum_{i=1}^I\left(\frac{1}{I-i+1}+\frac{1}{i}\right) \\
&= \sum_{i=1}^I\left(\frac{1}{T-t_i}+\frac{1}{t_i}\right)
\end{align*}
Time factor is larger in shared crossover case and so variance is larger
\end{frame}



\addtocounter{framenumber}{-1}
\begin{frame}{Delay in treatment effect on power [\hyperlink{slide:power}{S\ref{slide:power}}]} \label{apx:delay}
By [\hyperlink{apx:power}{A4}], larger $\Var(\hat{\theta})\Rightarrow$ lower power
\vspace{2em}

If $X_{ij}\in[0,1]$ and are known, an unbiased within-cluster estimator for $\theta$ is
\begin{align*}
\tilde{\theta} &= \left(\sum_{i=1}^I\sum_{j=1}^T\frac{X_{ij}}{T-t_i}\right)^{-1}\sum_{i=1}^I\left(\frac{\sum_{j=t_i+1}^T\bar{Y}_{ij.}}{T-t_i} - \frac{\sum_{j=1}^{t_i}\bar{Y}_{ij.}}{t_i}\right) \\
\Var(\tilde{\theta}) &= \frac{\sigma^2}{N}\left(\sum_{i=1}^I\sum_{j=1}^T\frac{X_{ij}}{T-t_i}\right)^{-2}\sum_{i=1}^I\left(\frac{1}{T-t_i} + \frac{1}{t_i}\right)
\end{align*}
If $X_{ij}\in(0,1)$ for some cluster $i$ and time $j$, then
\begin{align*}
 && &\sum_{i=1}^I\sum_{j=1}^T\frac{X_{ij}}{T-t_i} < \sum_{i=1}^I\sum_{j=t_i+1}^T\frac{1}{T-t_i} = I \\
 &\Rightarrow & & \frac{\sigma^2}{N}\left(\sum_{i=1}^I\sum_{j=1}^T\frac{X_{ij}}{T-t_i}\right)^{-2}\sum_{i=1}^I\left(\frac{1}{T-t_i} + \frac{1}{t_i}\right) > \frac{\sigma^2}{NI^2}\sum_{i=1}^I\left(\frac{1}{T-t_i} + \frac{1}{t_i}\right)
\end{align*}
RHS is variance of estimator with no delays and so delay increases variance
\end{frame}



\end{document}